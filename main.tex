\documentclass[]{article}

% Packages
\usepackage[utf8]{inputenc} % For UTF-8 encoding
\usepackage[T1]{fontenc}    % For proper Italian accents
\usepackage[italian]{babel} % Italian language support
\usepackage{amsmath, amsfonts, amssymb} % Math packages
\usepackage{graphicx}       % For including images
\usepackage{hyperref}       % For hyperlinks
\usepackage{bookmark}       % For improved PDF bookmarks and rerunfilecheck warning
\usepackage{tikz}           % For diagrams
\usepackage{geometry}       % For page layout
\usepackage{fancyhdr}       % For custom headers/footers
\usepackage{enumitem}       % For custom lists
\usepackage{xcolor}         % For colored text
\usepackage{mathtools}      % For additional math tools
\usepackage{amsthm}         % For theorem environments
\usepackage{quiver}        % For commutative diagrams

% Load custom commands
\usepackage{commands} 
% Theorem styles

% Page layout
\geometry{a4paper, margin=1in}
\pagestyle{fancy}
\fancyhf{}
\fancyhead[L]{Note del corso di Elementi di Topologia Algebrica}
\fancyhead[R]{\thepage}

\begin{document}

\title{Appunti di Elementi di Topologia Algebrica}
\author{Simone Riccio \\ \small{(dalle lezioni del Prof. Mario Salvetti)}}
\date{\today}

\maketitle

\tableofcontents

\section{Omologia Singolare}

\subsection{Omologia ridotta}
\begin{definition} [Complesso di catene aumentato] \nl
 Sia $X$ spazio topologico. Il \textbf{complesso di catene aumentato} $\tilde{C}(X)$ di $X$ è il complesso di catene:
 \[
    \cdots \xrightarrow{\partial_{n+1}} \tilde{C}_n(X) \xrightarrow{\partial_n} \tilde{C}_{n-1}(X) \xrightarrow{\partial_{n-1}} \cdots \xrightarrow{\partial_1} \tilde{C}_0(X) \xrightarrow{\epsilon} \mathbb{Z} \to 0
 \]
 dove $\epsilon: \tilde{C}_0(X) \to \mathbb{Z}$ è l'omomorfismo di aumentazione definito da:
 \[
    \epsilon\left(\sum_i n_i \sigma_i\right) = \sum_i n_i
 \]
 per ogni catena singolare $\sum_i n_i \sigma_i \in \tilde{C}_0(X)$.
\end{definition}

\begin{proposition}
 Sia $X$ spazio topologico, $H(X), \tilde{H(X)}$ rispettivamente l'omologia singolare e l'omologia singolare ridotta con coefficienti in $\mathbb{Z}$. \nl
 Vale:
 \[
    H_n(X) \cong \begin{cases}
        \tilde{H}_n(X) & n > 0 \\
        \tilde{H}_0(X) \oplus \mathbb{Z} & n = 0
    \end{cases}
 \]
\end{proposition}


\begin{proof}
    Il fatto di dimostra, applicando il teorema del morfismo di connessione alla successione esatta:
    \[
          0 \to \tilde{C}(X) \xrightarrow{} C(X) \xrightarrow{\epsilon} \mathbb{Z} \to 0
    \]
    Qui intendo con $\mathbb{Z}$ il complesso di catene che in grado 0 è $\mathbb{Z}$ e in tutti gli altri gradi è $\set{0}$. \nl
    Dunque si ha in omologia la successione esatta lunga:
    \[
        \cdots \to H_1(\mathbb{Z}) \cong \set{0} \to \tilde{H}_0(X) \to H_0(X) \to H_0(\mathbb{Z}) \cong \mathbb{Z} \to 0 \to \cdots
    \]
    e dato che la successione \textbf{scinde}, si ottiene l'isomorfismo voluto.
\end{proof}

\begin{remark} [Cosa significa che la successione scinde?] \nl
    Vedi teorema 102 del libro di Algebra 2. \nl
    Una successione esatta corta:
    \[
        0 \to A \xrightarrow{f} B \xrightarrow{g} C \to 0
    \]
    si dice che \textbf{scinde} se esiste un isomorfismo $\varphi: B \to A \oplus C$ tale che il seguente diagramma commuti:
    \[
        \begin{tikzcd}
            0 \arrow{r} & A \arrow{r}{f} \arrow[swap]{d}{\mathrm{id}_A} & B \arrow{r}{g} \arrow{d}{\varphi} & C \arrow{r} \arrow[swap]{d}{\mathrm{id}_C} & 0 \\
            0 \arrow{r} & A \arrow{r} & A \oplus C \arrow{r} & C \arrow{r} & 0
        \end{tikzcd}
    \]
    Ed è equivalente a dire che esiste un morfismo $s: C \to B$ tale che $g \circ s = \mathrm{id}_C$ (sezione destra) oppure un morfismo $r: B \to A$ tale che $r \circ f = \mathrm{id}_A$ (sezione sinistra).
    
\end{remark}

\subsection{Omotopia di catene e di mappe continue}
\begin{definition} [Omotopia di complessi di catene] \nl
        Siano $(C_*, \partial_*^C)$ e $(D_*, \partial_*^D)$ due complessi di catene e $f_\#, g_\#: C_* \to D_*$ due mappe di complessi di catene. \nl
        Un'\textbf{omotopia} tra $f_\#$ e $g_\#$ è una famiglia di omomorfismi di gruppi abeliani:
        \[
            p_q: C_q \to D_{q+1}
        \]
        tale che:
        \[
            \partial_{q}^D \circ p_q + p_{q-1} \circ \partial_q^C = g_q - f_q
        \]
\end{definition}

\begin{proposition} [Mappe di catene omotope di inducono la stessa mappa in omologia] \nl
    Siano $(C_*, \partial_*^C)$ e $(D_*, \partial_*^D)$ due complessi di catene e $f_\#, g_\#: C_* \to D_*$ due mappe di complessi di catene omotope. \nl
    Allora le mappe indotte $f_*, g_*: H_n(C_*) \to H_n(D_*)$ sono uguali per ogni $n \geq 0$.
\end{proposition}

\begin{proof} \nl
    Sia $[c] \in H_n(C_*)$ una classe di omologia, con $c \in Z_n(C_*)$. \nl
    Allora:
    \[
        g_n(c) - f_n(c) = \partial_n^D (p_n(c)) + p_{n-1}(\underbrace{\partial_n^C(c)}_{=0}) = \partial_n^D (p_n(c))
    \]
    Dunque $g_n(c)$ e $f_n(c)$ differiscono per un bordo, e quindi rappresentano la stessa classe di omologia in $H_n(D_*)$. \nl
    Quindi $g_*([c]) = f_*([c])$ per ogni $[c] \in H_n(C_*)$, da cui si conclude che $g_* = f_*$.
\end{proof}
\begin{definition} [Omotopia di mappe continue] \nl
    Siano $X, Y$ spazi topologici e $f, g: X \to Y$ due mappe continue. \nl
    Una \textbf{omotopia} tra $f$ e $g$ è una mappa continua:
    \[
        F: X \times I \to Y
    \]
    tale che:
    \[
        F(x, 0) = f(x), \quad F(x, 1) = g(x) \quad \forall x \in X
    \]
\end{definition}

\begin{theorem} [Invarianza omotopica dell'omologia singolare] \nl
    Siano $X, Y$ spazi topologici e $f, g: X \to Y$ due mappe continue omotope. \nl
    Allora le mappe indotte $f_*, g_*: H_n(X) \to H_n(Y)$ sono uguali per ogni $n \geq 0$. \nl
    Di conseguenza:
        \[X \sim Y \implies H_n(X) \cong H_n(Y) \text{ per ogni } n \geq 0.\]
\end{theorem}

\begin{proof} \nl
    \begin{enumerate}
        \item Se $F$ è un'omotopia tra $f$ e $g$, definiamo le mappe:
            \[
                \eta_s: X \to X \times I: x \mapsto (x, s) \quad s \in I
            \]
            che formano un'omotopia tra $\eta_0$ e $\eta_1$. \nl
            Si ha che $f = F \circ \eta_0$ e $g = F \circ \eta_1$. \nl
            Dunque se mostriamo che $(\eta_0)_* = (\eta_1)_*$, si ottiene per funtorialità che $f_* = g_*$.
        \item  Assumendo che due mappe omotope nel senso dei complessi di catene inducono la stessa mappa in omologia,
            dobbiamo mostrare che $(\eta_0)_\#$ e $(\eta_1)_\#*$ sono omotope nel senso dei complessi di catene. \nl
        Vogliamo costruire un'omotopia di complessi di catene:
            % https://q.uiver.app/#q=WzAsMTIsWzEsMCwiQ197cSsxfShYKSJdLFsyLDAsIkNfe3F9KFgpIl0sWzMsMCwiQ197cS0xfShYKSJdLFsxLDEsIkNfe3ErMX0oWSkiXSxbMiwxLCJDX3txfShZKSJdLFszLDEsIkNfe3EtMX0oWSkiXSxbNCwxXSxbNCwwXSxbMCwwXSxbMCwxXSxbMiw0LCJcXGJ1bGxldCJdLFsyLDMsIlxcYnVsbGV0Il0sWzAsMV0sWzEsMl0sWzMsNF0sWzUsNl0sWzQsNV0sWzIsN10sWzgsMF0sWzksM10sWzAsMywiXFxldGFfezAqfSIsMl0sWzAsMywiXFxldGFfezEqfSIsMCx7Im9mZnNldCI6LTN9XSxbMSw0LCJcXGV0YV97MCp9IiwyXSxbMSw0LCJcXGV0YV97MSp9IiwwLHsib2Zmc2V0IjotM31dLFsyLDUsIlxcZXRhX3swKn0iLDJdLFsyLDUsIlxcZXRhX3sxKn0iLDAseyJvZmZzZXQiOi0zfV0sWzEwLDExXV0=
            \[\begin{tikzcd}
                {} & {C_{q+1}(X)} & {C_{q}(X)} & {C_{q-1}(X)} & {} \\
                {} & {C_{q+1}(Y)} & {C_{q}(Y)} & {C_{q-1}(Y)} & {} \\
                \\
                \arrow[from=1-1, to=1-2]
                \arrow[from=1-2, to=1-3]
                \arrow["{\eta_{0\#}}"', from=1-2, to=2-2]
                \arrow["{\eta_{1\#}}", shift left=3, from=1-2, to=2-2]
                \arrow[from=1-3, to=1-4]
                \arrow["{\eta_{0\#}}"', from=1-3, to=2-3]
                \arrow["{\eta_{1\#}}", shift left=3, from=1-3, to=2-3]
                \arrow[from=1-4, to=1-5]
                \arrow["{\eta_{0\#}}"', from=1-4, to=2-4]
                \arrow["{\eta_{1\#}}", shift left=3, from=1-4, to=2-4]
                \arrow[from=2-1, to=2-2]
                \arrow[from=2-2, to=2-3]
                \arrow[from=2-3, to=2-4]
                \arrow[from=2-4, to=2-5]
            \end{tikzcd}\]
        Cioè una mappa $p: C_q(X) \to C_{q+1}(Y)$ che chiameremo \textbf{operatore prisma} tale che:
        \[
            \partial_{q-1}^Y \circ p_q + p_{q-1} \circ \partial_q^X = \eta_{1\#} - \eta_{0\#}
        \]
        \item Definiamo l'operatore prisma $p_q: C_q(X) \to C_{q+1}(Y)$ in maniera funtoriale come segue. \nl
        Se $p_q$ è funtoriale allora per ogni $f: X \to Y$ il seguente diagramma commuta:
        \[
        \begin{tikzcd}
            C_q(X) \arrow{r}{p_q} \arrow{d}{f_*} & C_{q+1}(X \times I) \arrow{d}{(f \times \mathrm{id}_I)_*} \\
            C_q(Y) \arrow{r}{p_q} & C_{q+1}(Y \times I)
        \end{tikzcd}
        \]
        Sia ora $\sigma: \Delta^q \to X$ un q-simplesso singolare di $X$. \nl
        Vale che $\sigma = \sigma_* (\mathrm{id}_{\Delta^q})$ \nl
        Di conseguenza per funtorialità vale:
        \[
        p_q(\sigma) = (\sigma \times \mathrm{id}_I)_* (p_q(\mathrm{id}_{\Delta^q}))
        \]
        Per cui basta definire $p_q(\mathrm{id}_{\Delta^q})$ per definire l'operatore prisma in generale. \nl
        \item Definiamo dunque $p_q(\mathrm{id}_{\Delta^q}): \Delta^{q+1} \to \Delta^q \times I$.
        Suddividiamo $\Delta^{q} \times I$ in $q+1$ simplessi $[v_0 \cdots v_i w_i \cdots w_q]$ per $i = 0, \ldots, q$, dove:
        \[
            \Delta^{q} \times \set{0} = [v_0 \cdots v_q], \quad \Delta^{q} \times \set{1} = [w_0 \cdots w_q]
        \]
        e l'operatore prisma è definito come:
        \[
            p_q(\mathrm{id}_{\Delta^q}) := \sum_{i=0}^q (-1)^i [v_0 \cdots v_i w_i \cdots w_q]
        \]
        Bisogna verificare che questa definizione soddisfi la proprietà richiesta:
        \[
            \partial_{q-1}^{X \times I} \circ p_q + p_{q-1} \circ \partial_q^X = \eta_{1\#} - \eta_{0\#}
        \]
        Si verifica calcolando i due termini a sinistra e sommando.
    \end{enumerate}
\end{proof}



\subsection{Relazione tra $H_1(X)$ e $\pi_1(X)$}

Per vedere i disegni di questa parte, che sono molto importanti, guardare le note di Simone Saccani. \nl
\begin{remark}
    Un cammino chiuso $\sigma: (I, \partial{I}) \to (X, x_0)$ può essere visto come $1$-simplesso signolare $\sigma: \Delta^1 \to X$ tale che $\partial{\sigma} = 0$. \nl
    Mostriamo che la classe di $\sigma$ in omologia di $H_1(X)$ dipende solo dalla classe di omotopia di $\sigma$ in $\pi_1(X, x_0)$ e che la giunzione di cammini corrisponde alla somma in omologia. \nl
    \begin{enumerate}
        \item Siano $\sigma \sim \sigma'$ lacci omotopi ad estremi fissi con omotopia $F: I \times I \to X$. \nl
        Dividiamo $I \times I$ in due simplessi: $[e_0 e_1 e_1']$ e $[e_0 e_1' e_0']$ con:
        \[
            e_0 = (0, 0), \quad e_1 = (1, 0), \quad e_1' = (1, 1), \quad e_0' = (0, 1)
        \]
        Consideriamo la catena singolare: $c = F \circ [e_0 e_1 e_1'] - F \circ [e_0 e_1' e_0']$. \nl
        Si verifica che $\partial{c} = \sigma' - \sigma$, dunque $[\sigma] = [\sigma']$ in $H_1(X)$.
        \item Sia $F: I \times I \to X$ una mappa continua tale che $F(t, s) = \sigma(t)$ per ogni $s \in I$. \nl
        A questo punto considerando la catena singolare $c = F \circ [e_0 e_1 e_1'] + F \circ [e_0 e_1' e_0']$ si verifica che $\partial{c} = \sigma + \sigma^{-1}$. Dunque $[\sigma^{-1}] = -[\sigma] +$ in $H_1(X)$.
        \item Infine vedere dalle dispense di Simone Saccani come si mostra che $[\sigma * \tau] = [\sigma] + [\tau]$ in $H_1(X)$.
    \end{enumerate}
\end{remark}

\begin{theorem} [Relazione tra $H_1(X)$ e l'abelianizzato di $\pi_1(X, x_0)$] \nl
    Sia $X$ spazio topologico con base puntata $x_0 \in X$. \nl
    Sia può definire $h': \pi_1(X, x_0) \to H_1(X)$ come:
    \[
        h'([\sigma]) = [\sigma]
    \]
    Poiché $H_1(X)$ è abeliano, $h'$ fattorizza attraverso l'abelianizzato di $\pi_1(X, x_0)$ dando luogo a un omomorfismo di gruppi:
    \[
        h: \pi_1(X, x_0)^{ab} \to H_1(X)
    \]
    Se $X$ è connesso per archi allora $h$ è un isomorfismo. Dunque:
    \[
        H_1(X) \cong \pi_1(X, x_0)^{ab}
    \]
\end{theorem}

\begin{proof}
    Si vuole costruire una inversa $l: H_1(X) \to \pi_1(X, x_0)^{ab}$. \nl
    Per ogni punto $x \in X$ indichiamo con $u_x$ un cammino da $x_0$ a $x$. \nl
    Definiamo 
    \[
        l': \begin{cases}
            C_1(X) \to \pi_1(X, x_0)^{ab} \\
            \sigma \mapsto [u_{\sigma(0)} * \sigma * u_{\sigma(1)}^{-1}]
        \end{cases}
    \]
    Per mostrare che $l'$ induce una mappa in omologia, dobbiamo verificare che $l'(\partial{c}) = 0$ per ogni $c \in C_2(X)$. \nl
    Un $2$-simplesso $\tau: \Delta^2 \to X$ definisce un'omotopia tra i cammini $\tau \circ [v_0 v_1]$, $\tau \circ [v_0 v_2]$ e $\tau \circ [v_1 v_2]$. \nl
    Si verifica che:
    \[
        l'(\partial{\tau}) = [u_{\tau(v_0)} * \tau \circ [v_1 v_2] * u_{\tau(v_2)}^{-1}] * [u_{\tau(v_0)} * \tau \circ [v_0 v_2] * u_{\tau(v_2)}^{-1}] * [u_{\tau(v_1)} * \tau \circ [v_0 v_1] * u_{\tau(v_0)}^{-1}] = 0
    \]
    Dunque $l'$ induce una mappa in omologia $l: H_1(X) \to \pi_1(X, x_0)^{ab}$. \nl
    Si verifica facilmente che $l$ e $h$ sono inverse l'una dell'altra.
\end{proof}

\subsection{Successione di Mayer-Vietoris}
\begin{definition} [Catene $\mathcal{U}$-piccole]
    Sia $\mathcal{U}$ un ricoprimento aperto di uno spazio topologico $X$. \nl
    Una catena $c = \sum_{\sigma} \nu_\sigma \sigma \in C_q(X)$ si dice \textbf{$\mathcal{U}$-piccola} se
    \[
        \forall \sigma \text{ con } \nu_\sigma \neq 0, \exists U \in \mathcal{U} : \mathrm{Im}(\sigma) \subseteq U.
    \]
    Si noti che se $c$ è $\mathcal{U}$-piccola, allora anche il suo bordo $\partial c$ è $\mathcal{U}$-piccolo. \nl
    Quindi le catene $\mathcal{U}$-piccole formano un sottocomplesso di catene di $C(X)$, che indichiamo con $C^{\mathcal{U}}(X)$.
\end{definition}

\begin{theorem} [Mayer-Vietoris] \nl
    Sia $X$ spazio topologico, $A, B \subset X$ tali che $X = \mathrm{Int}(A) \cup \mathrm{Int}(B)$. \nl
    Allora la successione esatta corta definita da 
    \[
        0 \to C_*(A \cap B) \xrightarrow{\varphi} C_*(A) \oplus C_*(B) \xrightarrow{\psi} C_*(X) \to 0
    \]
    dove $\varphi(c) = (c, c)$ e $\psi(a, b) = a - b$, induce in omologia la successione esatta lunga:
    \[
        \cdots \to H_n(A \cap B) \xrightarrow{\varphi_*} H_n(A) \oplus H_n(B) \xrightarrow{\psi_*} H_n(X) \xrightarrow{\Delta_n} H_{n-1}(A \cap B) \to \cdots
    \]
    con $\Delta: H(X) \to H(A \cap B)$ di grado $-1$.
\end{theorem}

\begin{proof} \nl
    Consideriamo il ricoprimento aperto $\mathcal{U} = \set{\mathrm{Int}(A), \mathrm{Int}(B)}$ di $X$. \nl
    \begin{enumerate}
        \item Dimostriamo prima assumendo che $H_q(X) \cong H^{\mathcal{U}}_q(X)$
        Vale per ogni $q$ che $C^{\mathcal{U}}_q(X) = C_q(A) + C_q(B)$, dunque la mappa $\psi: C_q(A) \oplus C_q(B) \to C^{\mathcal{U}}_q(X)$ definita da $\psi(c) = (a, b)$ con $c = a - b$, è suriettiva. \nl
        E dunque è esatta la successione:
        \[
            0 \to C_q(A \cap B) \xrightarrow{\varphi} C_q(A) \oplus C_q(B) \xrightarrow{\psi} C^{\mathcal{U}}_q(X) \to 0
        \]
        e per il teorema del morfismo di connessione si ottiene la successione esatta lunga in omologia:
        \[
            \cdots \to H_n(A \cap B) \xrightarrow{\psi_*} H_n(A) \oplus H_n(B) \xrightarrow{\varphi_*} H^{\mathcal{U}}_n(X) \xrightarrow{\Delta_n} H_{n-1}(A \cap B) \to \cdots
        \]
    \end{enumerate}
\end{proof}

\begin{proposition} [Omologia della sfera] \nl
    Sia $S^n$ la sfera n-dimensionale. \nl
    Allora:
    \[
        \tilde{H_q}(S^n) \cong \begin{cases}
            \mathbb{Z} & q = n \\
            0 & \text{altrimenti}
        \end{cases}
    \]
\end{proposition}

\begin{proof}
    Sia $S_n \cong A \cup B$ dove $A := S^n \setminus \{(0, \cdots, 1)\}, B := S^n \setminus \{(0, \cdots, -1)\}$. \nl
    $A, B$ sono contraibili, dunque $\tilde{H_q}(A) \cong \tilde{H_q}(B) \cong 0$ per ogni $q$. \nl
    Mentre $A \cap B$ si retrae per deformazione sull'equatore della sfera e dunque $\tilde{H_q}(A \cap B) \cong \tilde{H_q}(S^{n-1})$. \nl
    Applicando la successione di Mayer-Vietoris si ottiene la successione esatta lunga:
    \[
        \cdots \to \tilde{H_q}(S^{n-1}) \to 0 \to \tilde{H_q}(S^n) \to \tilde{H}_{q-1}(S^{n-1}) \to 0 \to \cdots
    \]

    Da cui si ottiene l'isomorfismo:
    \[
        \tilde{H_q}(S^n) \cong \tilde{H}_{q-1}(S^{n-1})
    \]

    Quindi dato che $S^0 = \set{0, 1}$ e dunque ha due componenti connesse, vale il caso base
    \[
        H_n(S^0) := \begin{cases}
            \mathbb{Z} & n = 0 \\
            0 & n \neq 0
        \end{cases}
    \] 
    da cui segue la tesi per induzione.
\end{proof}

\begin{theorem} [Punto fisso di Brower] \nl
    Sia $f: D^n \to D^n$ continua allora $\exists x \in D^n$ tale che $f(x) = x$ 
\end{theorem}

\begin{proof} \nl
    \begin{enumerate}
        \item Dimostriamo che non esiste una retrazione $r: D^n \to \partial{D^n} \cong S^{n-1}$. \nl
        Supponiamo per assurdo che esista tale retrazione. \nl
        Allora vale che l'identità su $S^{n-1}$ si fattorizza come:
        \[
            S^{n-1} \xhookrightarrow{i} D^n \xrightarrow{r} S^{n-1}
        \]
        dove $i$ è l'inclusione. \nl
        Per funtorialità in omologia si avrebbe che $(\mathrm{id}_{S^{n-1}})_*$ si fattorizza come:
        \[
            \mathbb{Z} \xhookrightarrow{i_*} 0 \xrightarrow{r_*} \mathbb{Z}
        \]
        assurdo.
        \item Supponiamo ora che esista una mappa continua $f: D^n \to D^n$ senza punti fissi. \nl
        Allora possiamo definire una retrazione $r: D^n \to S^{n-1}$ come:
        \[
            r(x) = \text{intersezione tra } S^{n-1} \text{ e la retta che congiunge } f(x) \text{ e } x
        \]
        
        Ma questo contraddice il punto precedente.
    \end{enumerate}
\end{proof}

\begin{proposition} \nl
    Sia $\mathrm{deg}: [S^n, S^n] \to \mathbb{Z}$ il grado di una mappa continua $f: S^n \to S^n$. \nl 
    Se $f: S^n \to S^n$ è senza punti fissi allora $f$ è omotopa alla mappa antipodale e quindi $\mathrm{deg}(f) = (-1)^{n+1}$.
\end{proposition}

\begin{proof}
    Supponiamo che $f$ sia senza punti fissi. \nl
    Allora possiamo definire un'omotopia tra $f$ e la mappa antipodale $a: S^n \to S^n$ come:
    \[
        F: S^n \times I \to S^n
    \]
    (Non mi è chiaro come continuare questa dimostrazione)
    Dunque per invarianza omotopica dell'omologia singolare vale che $\mathrm{deg}(f) = \mathrm{deg}(a) = (-1)^{n+1}$.
\end{proof}

\begin{theorem} [Palla pelosa] \nl
    Sia $v: S^n \to \R^{n+1}$ continua e tale che $v(x)$ sia ortogonale a $x$ e tangente ad $S^n$. \nl
\end{theorem}


\begin{definition} [Escissione] \nl
    Siano $U \subset A \subset X$ e l'inclusione $i: (X \setminus U, A \setminus U) \to (X, A)$. \nl
    Si dice che $U$ può essere \textbf{escissa} da $(X, A)$ se l'omomorfismo indotto in omologia:
    \[
        i_*: H_q(X \setminus U, A \setminus U) \to H_q(X, A)
    \]
    è un isomorfismo per ogni $q \geq 0$.
\end{definition}

\begin{theorem}
    [Teorema di escissione] \nl
        Siano $A \subset X$ e $\bar{U} \subset \mathrm{Int}(A)$. \nl
        Allora $U$ può essere escissa da $(X, A)$.
\end{theorem}

\begin{proof}
    Si definiscono due mappe di complessi di catene:
    \[
        \mathrm{sd}: C_q(X) \to C_q(X) \quad \text{e} \quad T: C_q(X) \to C_{q+1}(X)
    \] 
    osservando che per funtorialità basta definire $\mathrm{sd}(\mathrm{id}_{\Delta^q})$ e $T(\mathrm{id}_{\Delta^q})$. \nl
    Sia $B_q$ il baricentro di $\Delta^q$, e se $\sigma = [v_0 \cdots v_{q-1}]$ un $(q-1)$-simplesso allora $B_q \sigma = [B_q v_0 \cdots v_{q-1}]$ è il $q$-simplesso ottenuto aggiungendo il baricentro come vertice. \nl
    Si definiscono $\mathrm{sd}(\mathrm{id}_{\Delta^q})$ e $T(\mathrm{id}_{\Delta^q})$
    \[
        \mathrm{sd}(\mathrm{id}_{\Delta^q}) = \begin{cases}
            \mathrm{id}_{\Delta^0} & q = 0 \\
            B_q \mathrm{sd}(\partial{\Delta^q}) & q > 0
        \end{cases}
    \]
    \[
        T(\mathrm{id}_{\Delta^q}) = \begin{cases}
            0 & q = 0 \\
            B_q(\mathrm{id}_{\Delta^q} - \mathrm{sd}(\partial{\Delta^q}) - T(\partial{\Delta^q})) & q > 0
        \end{cases}
    \]
    Si verifica che valgono le seguenti proprietà:
    La suddivisione è un morfismo di complessi di catene:
    \[
        \partial_q \circ \mathrm{sd} = \mathrm{sd} \circ \partial_q
    \]
    mentre $T$ è un'omotopia di catene tra l'identità e la suddivisione:
    \[
        \partial_{q+1} \circ T_q + T_{q-1} \circ \partial_q = \mathrm{id}_{C_q(X)} - \mathrm{sd}_q
    \]
    Dunque $\mathrm{sd}_* \equiv \mathrm{id}_*$ in omologia. \nl

    \begin{lemma} \nl
    Sia $\sigma$ un $q$-simplesso geometrico, se $\tau$ è un simplesso in $sd(\sigma)$ allora
    \[
        \mathrm{diam}(\tau) \leq \frac{q}{q+1} \mathrm{diam}(\sigma)
    \]
    \end{lemma}

    \begin{proposition} \nl
        Sia $\mathcal{U} = \set{U_i}_{i \in I}$ un ricoprimento aperto di uno spazio topologico $X$. \nl
        Sia $\sigma: \Delta^q \to X$ un $q$-simplesso singolare, allora esiste $m_0 \in \N$ tale che 
        \[
            \forall m \geq m_0, \quad sd^m(\sigma) \in C_q^{\mathcal{U}}(X)
        \]
        Di conseguenza vale per ogni catena $c = \sum_{\sigma} \nu_\sigma \sigma \in C_q(X)$. \nl
    \end{proposition}

    \begin{theorem}
    Sia $\mathcal{U} = \set{U_i}_{i \in I}$ un ricoprimento aperto di uno spazio topologico $X$. \nl
    La mappa di inclusione $i: C_*^{\mathcal{U}}(X) \to C_*(X)$ induce un isomorfismo in omologia:
    \[
        i_*: H_q^{\mathcal{U}}(X) \to H_q(X)
    \]
    \end{theorem}

    \begin{proof}
        Mostriamo che la mappa di inclusione $i: C_*^{\mathcal{U}}(X) \to C_*(X)$ è surgettiva in omologia. \nl
        Se infatti $z \in C_q(X)$ è un ciclo, esiste $m_0$ tale che $sd^{m_0}(z) \in C_q^{\mathcal{U}}(X)$, ma poiché $sd_*$ è omotopa all'identità, si ha che $[z] = [sd^{m_0}(z)]$ in omologia. \nl
        Mostriamo ora che $i_*$ è iniettiva, se infatti $i_*([z]) = 0$ per un ciclo $z \in C_q^{\mathcal{U}}(X)$, esiste $c \in C_{q+1}(X)$ tale che $\partial{c} = z$. \nl
        \[
            [z] = [sd^{m_0}(z)] = [sd^{m_0}(\partial{c})] = [\partial{sd^{m_0}(c)}] = [0]
        \]
        in sostanza abbiamo dimostrato che se se una catena è un bordo in $C_*(X)$, allora è omotopa ad una catena $\mathcal{U}$-piccola ed è il bordo di una catena $\mathcal{U}$-piccola.
    \end{proof}

    \begin{corollary} \nl
        Sia $A \subset X$. Sia $C_q^{\mathcal{U}}(X, A) := C_q^{\mathcal{U}}(X) / C_q^{\mathcal{U}}(A)$ il complesso di catene relativo formato da catene $\mathcal{U}$-piccole. \nl
        La mappa di inclusione $i: C_*^{\mathcal{U}}(X, A) \to C_*(X, A)$ induce un isomorfismo in omologia:
        \[
            i_*: H_q^{\mathcal{U}}(X, A) \to H_q(X, A)
        \]
    \end{corollary}

    \begin{proof}
    Si considera il seguente diagramma commutativo con righe esatte:
    \[
        \begin{tikzcd}
            0 \arrow{r} & C_*^{\mathcal{U}}(A) \arrow{r} \arrow{d}{i_A} & C_*^{\mathcal{U}}(X) \arrow{r} \arrow{d}{i_X} & C_*^{\mathcal{U}}(X, A) \arrow{r} \arrow{d}{i} & 0 \\
            0 \arrow{r} & C_*(A) \arrow{r} & C_*(X) \arrow{r} & C_*(X, A) \arrow{r} & 0
        \end{tikzcd}
    \]
    Dalle ipotesi si ha che $i_A$ e $i_X$ inducono isomorfismi in omologia. \nl
     % https://q.uiver.app/#q=WzAsMTQsWzEsMCwiSF9xXntcXG1hdGhjYWx7VX19KEEpIl0sWzIsMCwiSF9xXntcXG1hdGhjYWx7VX19KFgpIl0sWzMsMCwiSF9xXntcXG1hdGhjYWx7VX19KFgsQSkiXSxbNCwwLCJIX3txLTF9XntcXG1hdGhjYWx7VX19KEEpIl0sWzUsMCwiSF97cS0xfV57XFxtYXRoY2Fse1V9fShYKSJdLFsxLDEsIkhfcShBKSJdLFsyLDEsIkhfcShYKSJdLFszLDEsIkhfcShYLCBBKSJdLFs0LDEsIkhfe3EtMX0oQSkiXSxbNSwxLCJIX3txLTF9KFgpIl0sWzAsMCwiXFxjZG90cyJdLFswLDEsIlxcY2RvdHMiXSxbNiwwLCJcXGNkb3RzIl0sWzYsMSwiXFxjZG90cyJdLFswLDVdLFswLDFdLFsxMCwwXSxbMTEsNV0sWzUsNl0sWzYsN10sWzEsMl0sWzIsM10sWzcsOF0sWzMsNF0sWzgsOV0sWzQsMTJdLFs5LDEzXSxbMSw2LCJpXyoiXSxbMyw4LCJpXyoiXSxbNCw5LCJpXyoiXV0=
        \[\begin{tikzcd}
            \cdots & {H_q^{\mathcal{U}}(A)} & {H_q^{\mathcal{U}}(X)} & {H_q^{\mathcal{U}}(X,A)} & {H_{q-1}^{\mathcal{U}}(A)} & {H_{q-1}^{\mathcal{U}}(X)} & \cdots \\
            \cdots & {H_q(A)} & {H_q(X)} & {H_q(X, A)} & {H_{q-1}(A)} & {H_{q-1}(X)} & \cdots
            \arrow[from=1-1, to=1-2]
            \arrow[from=1-2, to=1-3]
            \arrow["{i_*}"', "\cong", from=1-2, to=2-2]
            \arrow[from=1-3, to=1-4]
            \arrow["{i_*}"', "\cong", from=1-3, to=2-3]
            \arrow[from=1-4, to=1-5]
            \arrow["{i_*}"', from=1-4, to=2-4]
            \arrow[from=1-5, to=1-6]
            \arrow["{i_*}"', "\cong", from=1-5, to=2-5]
            \arrow[from=1-6, to=1-7]
            \arrow["{i_*}"', "\cong", from=1-6, to=2-6]
            \arrow[from=2-1, to=2-2]
            \arrow[from=2-2, to=2-3]
            \arrow[from=2-3, to=2-4]
            \arrow[from=2-4, to=2-5]
            \arrow[from=2-5, to=2-6]
            \arrow[from=2-6, to=2-7]
        \end{tikzcd}\]

    Dunque per il lemma del cinque anche $i_*$ è isomorfismo in omologia.
    \end{proof}

    \begin{definition} [Buona coppia]
        Una coppia $(X, A)$ di spazi topologici si dice \textbf{buona} se $A$ è chiuso ed è retratto di deformazione di un suo intorno in $X$.
    \end{definition}

    \begin{corollary} [Buona coppia] \nl
        Se $(X, A)$ è una buona coppia allora
        \[
            H_q(X, A) \cong \tilde{H}_q(\frac{X}{A})
        \]
    \end{corollary}

    \begin{proof} \nl
        % https://q.uiver.app/#q=WzAsNixbMCwwLCJIX3EoWCxBKSJdLFsxLDAsIkhfcShYLFUpIl0sWzIsMCwiSF9xKFggXFxzZXRtaW51cyBBLCBVIFxcc2V0bWludXMgQSkiXSxbMCwxLCJIX3EoWCAvIEEsIFtBXSkiXSxbMSwxLCJIX3EoWCBcXHNldG1pbnVzIEEsIFxccGkoVSkpIl0sWzIsMSwiSF9xKFxcZnJhY3tYfXtBfVxcc2V0bWludXMgW0FdLCBcXHBpKFUpIFxcc2V0bWludXMgW0FdKSJdLFswLDEsIlxcY29uZyJdLFsxLDIsIlxcY29uZyJdLFszLDQsIlxcY29uZyJdLFs0LDUsIlxcY29uZyJdLFsyLDUsIlxcY29uZyIsMl0sWzAsMywiXFxjb25nIiwyLHsic3R5bGUiOnsiYm9keSI6eyJuYW1lIjoiZGFzaGVkIn19fV1d
    \[\begin{tikzcd}
        {H_q(X,A)} & {H_q(X,U)} & {H_q(X \setminus A, U \setminus A)} \\
        {H_q(X / A, [A])} & {H_q(X \setminus A, \pi(U))} & {H_q(\frac{X}{A}\setminus [A], \pi(U) \setminus [A])}
        \arrow["\cong", from=1-1, to=1-2]
        \arrow["\cong"', dashed, from=1-1, to=2-1]
        \arrow["\cong", from=1-2, to=1-3]
        \arrow["\cong"', from=1-3, to=2-3]
        \arrow["\cong", from=2-1, to=2-2]
        \arrow["\cong", from=2-2, to=2-3]
    \end{tikzcd}\]
        Le frecce orizzontali a sinistra sono isomorfismi perché $U$ e $A$ sono omotopicamente equivalenti, mentre quelle a destra per il teorema di escissione. \nl
        Ci sarebbe da verificare che $\pi(U)$ è un intorno di $[A]$ in $X / A$ e che $[A]$ è retratto di deformazione di $\pi(U)$. \nl
        L'isomorfismo verticale a destra si ha perché la proiezione al quoziente su $X \setminus A$ è un omeomorfismo dato che stiamo togliendo $A$.
        Ne segue che esiste un isomorfismo verticale a sinistra. \nl
        Infine si nota che per ogni spazio topologico $X$ vale
        \[
            H_q(X, x_0) \cong \tilde{H}_q(X)
        \]
    \end{proof}

    \begin{theorem} [Lemma dei 5] \nl
        Sia il seguente diagramma commutativo con righe esatte:
        \[
            \begin{tikzcd}
                A_1 \arrow{r} \arrow{d}{\alpha_1} & A_2 \arrow{r} \arrow{d}{\alpha_2} & A_3 \arrow{r} \arrow{d}{\alpha_3} & A_4 \arrow{r} \arrow{d}{\alpha_4} & A_5 \arrow{d}{\alpha_5} \\
                B_1 \arrow{r} & B_2 \arrow{r} & B_3 \arrow{r} & B_4 \arrow{r} & B_5
            \end{tikzcd}
        \]
        Se $\alpha_1, \alpha_2, \alpha_4, \alpha_5$ sono isomorfismi allora anche $\alpha_3$ è un isomorfismo.
    \end{theorem}

    Per finire la dimostrazione del teorema di Escissione. consideriamo $\mathcal{U} = \set{X \setminus U, (A)}$ un ricoprimento di $X$. \nl
    $\mathcal{U}$ non è un ricoprimento aperto, ma contiene un ricoprimento aperto dato da $\mathcal{V} = \set{X \setminus \bar{U}, \mathrm{Int}(A)}$. \nl
    Consideriamo:
   
    % https://q.uiver.app/#q=WzAsNCxbMCwwXSxbMSwwLCJDX3EoWCBcXHNldG1pbnVzIEEsIEEgXFxzZXRtaW51cyBVKSJdLFsyLDAsIkNfcV57XFxtYXRoY2Fse1V9fShYLCBBKSJdLFszLDAsIkNfcShYLCBBKSJdLFsxLDIsImoiXSxbMiwzLCJpIiwwLHsic3R5bGUiOnsidGFpbCI6eyJuYW1lIjoiaG9vayIsInNpZGUiOiJ0b3AifX19XV0=
    \[\begin{tikzcd}
        {} & {C_q(X \setminus A, A \setminus U)} & {C_q^{\mathcal{U}}(X, A)} & {C_q(X, A)}
        \arrow["j", from=1-2, to=1-3]
        \arrow["i", hook, from=1-3, to=1-4]
    \end{tikzcd}\]

    Se dimostriamo che $j$ è un isomorfismo di complessi di catene, allora dato che per il teorema precedente $i$ induce un isomorfismo in omologia, si conclude che l'inclusione $j \circ i: C_q(X \setminus U, A \setminus U) \to C_q(X, A)$ induce un isomorfismo in omologia. \nl

    \begin{enumerate}
        \item Mostriamo che $j$ è surgettiva. \nl
        Sia $z \in C_q^{\mathcal{U}}(X)$ allora $z = z' + z''$ con $z' \in C_q(X \setminus U)$ e $z'' \in C_q(A)$. \nl
        Ma quindi $[z] = [z' + z''] = [z']$ in $C_q^{\mathcal{U}}(X, A)$, dunque $j([z']) = [z]$.
        \item Mostriamo che $j$ è iniettiva. \nl
        Sia $z \in C_q(X \setminus U, A \setminus U)$ tale che $j([z]) = 0$ allora $z \in C_q^{\mathcal{U}}(A) = $, dunque, poiché $z \in C_q(X \setminus U)$ sarà nell'intersezione, cioè $z \in C_q(A \setminus U)$ e quindi $[z] = 0$ in $C_q(X \setminus U, A \setminus U)$.
    \end{enumerate}
\end{proof}

\begin{definition} (Coppia escissiva) \nl
    Una coppia $(X_1, X_2)$ di spazi topologici si dice \textbf{escissiva} se le inclusioni:
    \[
        i_1: (X_1, X_1 \cap X_2) \hookrightarrow (X_1 \cup X_2, X_1 \cup X_2), \quad i_2: (X_2, X_1 \cap X_2) \hookrightarrow (X_1 \cup X_2, X_1)
    \]
    inducono isomorfismi in omologia.
\end{definition}

\begin{lemma} \nl
    Sia $(X_1, X_2)$ una coppia di spazi e $i: C_q(X_1) + C_q(X_2) \to C_q(X_1 \cup X_2)$ l'inclusione. \nl 
    \[
        (X_1, X_2) \text{ escissiva } \iff i_*: H_q(X_1) + H_q(X_2) \to H_q(X_1 \cup X_2) \text{ è un isomorfismo}
    \]
\end{lemma}

\begin{proof}
    Considero la successione esatta corta:
    % https://q.uiver.app/#q=WzAsNSxbMCwwLCIwIl0sWzEsMCwiQ19xKFhfMSkiXSxbMiwwLCJDX3EoWF8xKStDX3EoWF8yKSJdLFszLDAsIlxcbGVmdChDX3EoWF8xKStDX3EoWF8yKVxccmlnaHQpL3tDX3EoWF8xIFxcY2FwIFhfMil9Il0sWzQsMCwiMCJdLFswLDFdLFsxLDIsImkiXSxbMiwzLCJcXHBpIl0sWzMsNF1d
    \[\begin{tikzcd}
        0 & {C_q(X_1)} & {C_q(X_1)+C_q(X_2)} & {\left(C_q(X_1)+C_q(X_2)\right)/{C_q(X_1 \cap X_2)}} & 0
        \arrow[from=1-1, to=1-2]
        \arrow["i", from=1-2, to=1-3]
        \arrow["\pi", from=1-3, to=1-4]
        \arrow[from=1-4, to=1-5]
    \end{tikzcd}\]
    Ma vale per il secondo teorema di isomorfismo
    \[
        \left(C_q(X_1) + C_q(X_2)\right)/C_q(X_1 \cap X_2) \cong C_q(X_2) / C_q(X_1 \cap X_2) 
    \]
    quindi
     \[\begin{tikzcd}
        0 & {C_q(X_1)} & {C_q(X_1)+C_q(X_2)} & {C_q(X_2)/C_q(X_1 \cap X_2)} & 0
        \arrow[from=1-1, to=1-2]
        \arrow["i", from=1-2, to=1-3]
        \arrow["\pi", from=1-3, to=1-4]
        \arrow[from=1-4, to=1-5]
    \end{tikzcd}\]
    Se verificassimo che il seguente diagramma sia commutativo

    % https://q.uiver.app/#q=WzAsMTQsWzAsMCwiSF97cSsxfShYXzIsIFhfMSBcXGNhcCBYXzIpIl0sWzAsMSwiSF9xKFhfMSBcXGN1cCBYXzIsIFhfMSkiXSxbMSwwLCJIX3EoWCkiXSxbMiwwLCJIX3EoQ18qKFhfMSkrQ18qKFhfMikpIl0sWzMsMCwiSF9xKFhfMiwgWF8xIFxcY2FwIFhfMikiXSxbNCwwLCJIX3txLTF9KFhfMSkiXSxbMSwxLCJIX3EoWCkiXSxbMiwxLCJIX3EoWF8xIFxcY3VwIFhfMikiXSxbMywxLCJIX3EoWF8xIFxcY3VwIFhfMiwgWF8xKSJdLFs0LDEsIkhfe3EtMX0oWF8xKSJdLFs1LDAsIkhfcShDXyooWF8xKStDXyooWF8yKSkiXSxbNSwxLCJIX3EoWF8xIFxcY3VwIFhfMikiXSxbNiwwLCJcXGNkb3RzIl0sWzYsMSwiXFxjZG90cyJdLFsyLDNdLFswLDJdLFszLDRdLFs0LDVdLFs1LDEwXSxbNiw3XSxbNyw4XSxbOCw5XSxbOSwxMV0sWzUsOSwiXFxtYXRocm17aWR9XyoiLDJdLFs0LDgsIlxccHNpIiwyXSxbMyw3LCJcXHZhcnBoaSIsMl0sWzIsNiwiXFxtYXRocm17aWR9XyoiLDJdLFsxLDZdLFsxMCwxMl0sWzExLDEzXSxbMTAsMTEsIlxcdmFycGhpIiwyXSxbMCwxLCJcXHBzaSIsMl1d
\[\begin{tikzcd}[column sep=tiny]
	{H_{q+1}(X_2, X_1 \cap X_2)} & {H_q(X)} & {H_q(C_*(X_1)+C_*(X_2))} & {H_q(X_2, X_1 \cap X_2)} & {H_{q-1}(X_1)} & {H_q(C_*(X_1)+C_*(X_2))} & \cdots \\
	{H_q(X_1 \cup X_2, X_1)} & {H_q(X)} & {H_q(X_1 \cup X_2)} & {H_q(X_1 \cup X_2, X_1)} & {H_{q-1}(X_1)} & {H_q(X_1 \cup X_2)} & \cdots
	\arrow[from=1-1, to=1-2]
	\arrow["\psi"', from=1-1, to=2-1]
	\arrow[from=1-2, to=1-3]
	\arrow["{\mathrm{id}_*}"', from=1-2, to=2-2]
	\arrow[from=1-3, to=1-4]
	\arrow["\varphi"', from=1-3, to=2-3]
	\arrow[from=1-4, to=1-5]
	\arrow["\psi"', from=1-4, to=2-4]
	\arrow[from=1-5, to=1-6]
	\arrow["{\mathrm{id}_*}"', from=1-5, to=2-5]
	\arrow[from=1-6, to=1-7]
	\arrow["\varphi"', from=1-6, to=2-6]
	\arrow[from=2-1, to=2-2]
	\arrow[from=2-2, to=2-3]
	\arrow[from=2-3, to=2-4]
	\arrow[from=2-4, to=2-5]
	\arrow[from=2-5, to=2-6]
	\arrow[from=2-6, to=2-7]
\end{tikzcd}\]

A questo punto, se la coppia è escissiva allora $\psi$ è isomorfismo e $\varphi$ lo è per il lemma dei 5. \nl
Se vale $RHS$ allora $\varphi$ è isomorfismo e quindi $\psi$ è isomorfismo sempre per il lemma dei 5.
\end{proof}

\begin{proof} (fine della dimostrazione di Mayer Vietoris) \nl
     
\end{proof}

\subsection{Teoria omologica}
\begin{definition} (Categoria ammissibile) \nl
    Data una categoria $\mathcal{C}$, con oggetti $\mathcal{A}$ di coppie di spazi, si dice che $\mathcal{C}$ è \textbf{amissibile}
    \begin{enumerate}
        \item $(X, A) \in \mathcal{A} \implies (X, X), (A, A), (X, \emptyset), (A, \emptyset) \in \mathcal{A}$
        \item $(X,A) \in \mathcal{A} \implies (X \times I, A \times I) \in \mathcal{A}$
        \item $(P, \emptyset) \in \mathcal{A}$ per ogni $P$ punto
    \end{enumerate}
\end{definition}

\begin{definition} (Teoria omologica) \nl
    Data una categoria $\mathcal{C}$ \textbf{ammissibile}, con oggetti $\mathcal{A}$ di coppie di spazi, una \textbf{teoria omologica} su $\mathcal{C}$ è il dato di:
    \begin{enumerate}
        \item Una famiglia di funtori covarianti $H_q: \mathcal{C} \to \mathrm{Ab}$ per ogni $q \geq 0$
        \item Se $(X, A) \in \mathcal{A}$, $H_q$ è naturale. (da aggiustare)
    \end{enumerate}
    tali che valgono i seguenti assiomi:
    \begin{enumerate}
        \item (Successione della coppia) Per ogni $(X, A) \in \mathcal{A}$ esiste una successione esatta lunga in omologia:
        \[
            \cdots \to H_q(A) \xrightarrow{i_*} H_q(X) \xrightarrow{j_*} H_q(X, A) \xrightarrow{\Delta_q} H_{q-1}(A) \to \cdots
        \]
        \item (Invarianza omotopica) Siano $f, g: (X, A) \to (Y, B)$ due mappe omotope in $\mathcal{C}$. \nl
        Allora $H_q(f) = H_q(g): H_q(X, A) \to H_q(Y, B)$ per ogni $q \geq 0$
        \item  (Escissione) Sia $(X, A) \in \mathcal{A}$ e $U \subset A$ tale che $\bar{U} \subset \mathrm{Int}(A)$. \nl
        Allora l'inclusione $i: (X \setminus U, A \setminus U) \to (X, A)$ induce un isomorfismo in omologia:
        \[  
            i_*: H_q(X \setminus U, A \setminus U) \to H_q(X, A)
        \]
        per ogni $q \geq 0$.
        \item (Dimensionalità) Sia $P$ un punto, allora $H_q(P) = 0$ per ogni $q > 0$.
        \item (Coppia compatta)
    \end{enumerate}
\end{definition}

\begin{theorem} (Unicità della teoria omologica) \nl
    Esiste un'unica teoria omologica su $\mathcal{C} = {X | \text{supporti di complessi simplicali}}$ che soddisfa gli assiomi sopra elencati, ed è l'omologia singolare.
    Dunque i gruppi di omologia singolare, omologia simpliciale e omologia delta-simplicale sono isomorfi.
\end{theorem}

\subsection{Jordan e Invarianza del dominio}

\begin{definition} 
    Uno spazio topologico $X \cong D^n$ si dice \textbf{n-cella}
\end{definition}

\begin{definition}
    Uno spazio topologico $X$ si dice \textbf{aciclico} se $\tilde{H_q}(X) = 0$ per ogni $q \geq 0$.
\end{definition}

\begin{definition}
    Dato $A \subset X$ si dice che $A$ \textbf{separa} $X$ se $X \setminus A$ è sconnesso.
\end{definition}

\begin{remark} \nl
    $A \subset X$ chiuso e $X$ localmente connesso per archi. \nl
    \[A \text{ separa } X \iff \tilde{H_0}(X \setminus A) \neq 0\].
\end{remark}

\begin{theorem}  \nl
    Se $B \subset S^n$ una k-cella, allora $S^n \setminus B$ è aciclico. 
\end{theorem}

\begin{proof} \nl
    Poiché $I^k \cong D^k$, consideriamo $B \cong I^k \subset S^n$ e dimostriamo per induzione su $k$.
    \begin{enumerate}
        \item Caso base $k = 0$. \nl
        Allora $B = \set{x_0}$ è un punto e dunque $S^n \setminus B \cong \R^n$ è aciclico.
        \item Passo induttivo. \nl
        Supponiamo che la tesi sia vera per $k-1$. \nl
        Consideriamo $B = I^k = I^{k-1} \times I \subset S^n$ e sia $h: I^k \hookrightarrow S^n$. \nl
        Siano $B_1 = h(I^{k-1} \times [0, 1/2))$ e $B_2 = h(I^{k-1} \times (1/2, 1])$. \nl
        Se $C = B_1 \cap B_2 = h(I^{k-1} \times \{\frac{1}{2}\})$, vale:
        \[
            (X \setminus B_1) \cup (X \setminus B_2) = X \setminus B  
        \]
        \[
            (X \setminus B_1) \cap (X \setminus B_2) = X \setminus C
        \]
        si ha dunque per Mayer-Vietoris la successione esatta lunga:
        

        da finire.

        \end{enumerate}
\end{proof}

\begin{theorem} \nl
    Dato $0 \leq k < n$, se $h: S^k \hookrightarrow S^n$ è un'immersione continua allora l'immagine $h(S^k)$ separa $S^n$, in particolare:
    \[
        \tilde{H_q}(S^n \setminus h(S^k)) = \begin{cases}
            \mathbb{Z} & q = n - k - 1 \\
            0 & \text{altrimenti}
        \end{cases}
    \]
\end{theorem}

\begin{proof} \nl
    Sia $S^k = E_{+}^k \cup E_{-}^k$ dove $E_{+}^k$ e $E_{-}^k$ sono le due k-celle che compongono $S^k$. \nl
    Vale che $E_{+}^k \cap E_{-}^k \sim S^{k-1}$(si retraggono per deformazione sull'equatore). \nl
    Siano $A = S^n \setminus h(E_{+}^k)$ e $B = S^n \setminus h(E_{-}^k)$. \nl
    Vale che $A \cap B = S^n \setminus h(S^k)$ e $A \cup B = S^n \setminus h(S^{k-1})$. \nl
    Per Mayer-Vietoris si ha la successione esatta lunga:
   % https://q.uiver.app/#q=WzAsNyxbMSwwLCJcXHRpbGRle0hfe3F9fVxcbGVmdChTXm4gXFxzZXRtaW51cyBoXFxsZWZ0KFNea1xccmlnaHQpXFxyaWdodCkiXSxbMiwwLCIwIl0sWzMsMCwiXFx0aWxkZXtIX3txfX1cXGxlZnQoU15uIFxcc2V0bWludXMgaFxcbGVmdChTXntrLTF9XFxyaWdodClcXHJpZ2h0KSJdLFs0LDAsIlxcdGlsZGV7SH1fe3EtMX1cXGxlZnQoU15uIFxcc2V0bWludXMgaFxcbGVmdChTXmtcXHJpZ2h0KVxccmlnaHQpIl0sWzAsMCwiXFxjZG90cyJdLFs1LDAsIjAiXSxbNiwwLCJcXGJ1bGxldCJdLFswLDFdLFsxLDJdLFsyLDNdLFs0LDBdLFszLDVdLFs1LDYsIlxcY2RvdHMiXV0=
\[\begin{tikzcd}
	\cdots & {\tilde{H_{q}}\left(S^n \setminus h\left(S^k\right)\right)} & 0 & {\tilde{H_{q}}\left(S^n \setminus h\left(S^{k-1}\right)\right)} & {\tilde{H}_{q-1}\left(S^n \setminus h\left(S^k\right)\right)} & 0 & \bullet
	\arrow[from=1-1, to=1-2]
	\arrow[from=1-2, to=1-3]
	\arrow[from=1-3, to=1-4]
	\arrow[from=1-4, to=1-5]
	\arrow[from=1-5, to=1-6]
	\arrow["\cdots", from=1-6, to=1-7]
\end{tikzcd}\]

    $\tilde{H_{q}}(A) \oplus \tilde{H_{q}}(B) = 0$ perché $A$ e $B$ sono aciclici (per il teorema precedente). \nl
    Quindi per esattezza vale:
    \[
        H_{q}(S^n \setminus h(S^{k-1})) \cong H_{q-1}(S^n \setminus h(S^k))
    \]
    Da questo punto:
    \[
        \tilde{H_q}(S^n \setminus h(S^k)) \cong \tilde{H}_{q+1}(S^n \setminus h(S^{k-1})) = \dots = \tilde{H}_{q+k}(S^n \setminus h(S^0)) = \begin{cases}
            \mathbb{Z} & q + k = n - 1 \\
            0 & \text{altrimenti}
        \end{cases}
    \]
\end{proof}

\begin{theorem} [Separazione di Jordan-Brouwer] \nl
    Sia $h: S^{n-1} \hookrightarrow S^n$ un'immersione continua. \nl
    Allora l'immagine $h(S^{n-1})$ separa $S^n$ in due componenti connesse per archi $K_1, K_2$. \nl
    Inoltre $\partial{K_1}, \partial{K_2} = h(S^{n-1})$
\end{theorem}

\begin{proof}
    La prima parte segue dal teorema precedente immediatamente dato che 
    \[
        \tilde{H}_0(S^n \setminus h(S^{n-1})) = \mathbb{Z}
    \]
    poiché $n - (n-1) - 1 = 0$. \nl
    La parte che resta da dimostrare è che i bordi delle due componenti sono uguali a $h(S^{n-1})$. \nl
    Innanzitutto vale $\partial{K_1}, \partial{K_2} \subset h(S^{n-1})$ perché vale $\partial K_1 \cap K_2 = K_1 \cup \partial K_2$. \nl
    L'altro contenimento è più complicato. \nl
    Sia $x \in h(S^{n-1})$ e mostriamo che per ogni $U \subset S^n$ un intorno aperto di $x$, si ha $U \cap h(S^{n-1}) \neq \emptyset$. \nl
    Sia $A \subset U$ intorno aperto di $x$ tale che $h(S^{n-1}) \setminus A  \cong D^{n-1}$. \nl
    Intanto $S^n \setminus (h(S^{n-1}) \setminus A)$ è connesso per archi per il lemma precedente. \nl
    A questo punto se senza perdita di generalità supponiamo che $A \cap K_2 = \emptyset$, potrei scrivere:
    \[
        S^n \setminus (h(S^{n-1}) \setminus A) = (K_1 \cup A) \cup K_2
    \]
    Ma questo è assurdo perché il lato sinistro è connesso per archi mentre il lato destro no. \nl
    Dunque $A$ interseca entrambe le componenti $K_1$ e $K_2$ e quindi $x \in \partial K_1$ e $x \in \partial K_2$.

\end{proof}

\begin{example} (Sfera cornuta di Alexander)
    Esiste un'immersione continua $h: S^2 \hookrightarrow \R^3$ tale che l'immagine $h(S^2)$ ha complementare non semplicemente connesso.
\end{example}

\begin{corollary} \nl
    Se $h: S^{n-1} \hookrightarrow \R^n$ è un'immersione continua allora continua a valere la tesi del teorema di separazione di Jordan-Brower.
\end{corollary}

\begin{corollary} (Teorema classico della curva di Jordan) \nl
    Sia $h: S^1 \hookrightarrow \R^2$ un'immersione continua. \nl
    Allora l'immagine $h(S^1)$ separa $\R^2$ in due componenti connesse per archi $K_1, K_2$. \nl
    Inoltre $\partial{K_1}, \partial{K_2} = h(S^1).$ \nl
    $K_1$ è limitata ed è detto \textbf{interno} e $K_2$ non lo è ed è detto \textbf{esterno}.
\end{corollary}

\subsection{Classificazione delle 2-varietà}

\begin{definition} [Omologia locale]
    Sia $X$ uno spazio topologico e $x \in X$. \nl
    La \textbf{omologia locale} di $X$ in $x$ è definita come:
    \[
        H_q(X, X \setminus x) := H_q(X, X \setminus \set{x})
    \]
\end{definition}

\begin{lemma}
    Sia $X$ uno spazio topologico e $A$ è un intorno di $x$. \nl
    Allora vale:
    \[
        H_q(X, X \setminus \set{x}) \cong H_q(A, A \setminus \set{x})
    \]
\end{lemma}

\begin{proof} \nl
    Si può dimostrare usando l'escissione considerando $U = X \setminus A$. \nl
    A quel punto poiché $\bar{U} \subset X \setminus \set{x}$(vero perché $A$ è intorno di $x$), si ha che l'inclusione:
    \[
        i: (A, A \setminus \set{x}) \hookrightarrow (X, X \setminus \set{x})
    \]
    induce un isomorfismo in omologia.
\end{proof}

\begin{remark}
    Data la coppia $(A, B)$ e $i: C \hookrightarrow B$ retratto di deformazione di $B$,
    si ha un isomorfismo in omologia:
    \[
        H_q(A, B) \cong H_q(A, C)
    \]
    Segue dal lemma dei 5 e dalla successione della coppia.
    Questo permette di ricavare il teorema successivo in maniera ovvia.
\end{remark}

\begin{theorem}
    Siano $U \subset \R^n$ e $V \subset \R^m$ entrambi aperti.
    \[
        U \cong V \iff n = m
    \]
\end{theorem}

\begin{definition} [n-varietà topologica] \nl
    Uno spazio topologico $X$ di Hausdorff si dice \textbf{n-varietà topologica} se per ogni $x \in X$ esiste un intorno aperto $U$ di $x$ e $V \subset \R^n$ aperto tali che 
    \[x \in U \cong V \subset \R^n\].
\end{definition}

\section{Omologia cellulare}

\subsection{CW-complessi}
\begin{definition} [CW-complessi]
    Si dice che uno spazio topologico $X$ ha struttura di \textbf{CW-complesso} se può essere cosrtruito nella seguente maniera:
    \begin{enumerate}
        \item Si parte da un insieme discreto di punti $X_0$ detto \textbf{0 scheletro}.
        \item Si costruisce il \textbf{n-scheletro} $X_n$ a partire da $X_{n-1}$ nella seguente maniera: \nl
         \[
         X_n = X_{n-1} \cup \bigcup_{\alpha \in S_n} \bar{e_\alpha^n}  
         \]
         dove $e_\alpha^n$ sono n-celle attaccate tramite mappe continue
         \[\varphi_\alpha^n: \partial D^n \cong S^{n-1} \to X_{n-1}\]
         dette \textbf{funzioni di attaccamento}, che identificano il bordo della n-cella con una parte di $X_{n-1}$.
         La funzione di attaccamento si può estendere ad una funzione continua:
         \[
            \Phi_\alpha^n: D^n \to X_n
        \]
        detta \textbf{funzione caratteristica}, tale che
        \[
        \Phi_\alpha^n\big|_{\mathrm{Int}(D^n)}:\mathrm{Int}(D^n)\xrightarrow{\ \cong\ } e_\alpha^n,
        \]
        ovvero $\Phi_\alpha^n$ ristretto a $\mathrm{Int}(D^n)$ è un omeomorfismo sulla sua immagine $e_\alpha^n$.
    \end{enumerate}
\end{definition}

\begin{remark} \nl
    $\varphi_\alpha(\partial D^n) \subset X_{n-1}$ può non essere una $(n-1)$-cella. \nl 
    Se vale che $\varphi_\alpha(\partial D^n)$ è una $(n-1)$-cella e $\varphi_\alpha$ sono tutti iniettivi, allora $X$ si dice essere un \textbf{CW-complesso regolare}.
\end{remark}

\begin{proposition} \nl
    Sia $X$ un CW-complesso, se $K \subset X$ è compatto allora $K$ è contenuto in un sottocomplesso di $X$ finitio. 
\end{proposition}

\begin{proof} \nl
    Si supponga per assurdo che esista una successione infinita $S = {x_i}_{i \in \N} \subset K$ tale che
    ogni $x_i$ appartiene a una cella diversa. \nl
    Assumendo che $S \cap X_{n-1}$ sia chiuso si può dimostrare che $S \cap X_n$ è chiuso. \nl
    Infatti $\varphi^{-1}(S) \subset \partial D^n$ è chiuso perché $\partial D^n$ è compatto, ma allora lo è anche $\Phi^{-1}(S) \subset D^n$ poiché al più stiamo aggiungendo un punto ad un chiuso. \nl
    Quindi $S$ è chiuso in $X$. Allo stesso modo si dimostra che questo vale per ogni sottoinsieme di $S$, dunque $S$ ha la topologia discreta. \nl
    Ma questo è assurdo perché $S$ compatto con la topologia discreta implica che $S$ è finito. \nl
    Si è mostrato che $K$ è contenuto in un numero finito di celle, va mostrato che un numero finito di celle è contenuto in un sottocomplesso finitio. \nl
    Poiché unione finita di sottocomplessi finiti è un sottocomplesso finito, basta mostrare che ogni cella è contenuta in un sottocomplesso finito. \nl
    Sia dunque $e_\alpha^n$ una n-cella, per costruire un sottocomplesso finito che la contenga, si consideri l'immagine della funzione di attaccamento $\varphi_\alpha^n(\partial D^n) \subset X_{n-1}$. \nl
    Poiché $\partial D^n$ è compatto, per l'ipotesi iniziale esiste un sottocomplesso finito $Y_{n-1} \subset X_{n-1}$ che contiene $\varphi_\alpha^n(\partial D^n)$. \nl
    Si può dunque costruire un sottocomplesso finito $Y_n = Y_{n-1} \cup \bar{e_\alpha^n}$ che contiene $e_\alpha^n$.
\end{proof}

\begin{proposition} \nl
    I CW-complessi sono spazi normali, in particolare Hausdorff.
\end{proposition}

\begin{proof} \nl
    Innanzitutto è facile mostrare che i punti sono chiusi, dato che vengono rimandati tramite $\Phi_\alpha^n$ a punti di $D^n$ che sono chiusi. \nl
    Supponiamo ora di avere due insiemi chiusi disgiunti $A, B \subset X$. \nl
\end{proof}

\begin{theorem} [Definizione alla Whitehead di CW-complesso]
    $X$ spazio topologico di Hausdorff è un CW-complesso se e solo se data esiste una famiglia $\Phi_\alpha: D^n_\alpha \to X$ di mappe continue tali che:
    \begin{enumerate}
        \item \[
            \Phi_\alpha^n\big|_{\mathrm{Int}(D^n)}:\mathrm{Int}(D^n)\xrightarrow{\ \cong\ } e_\alpha^n,
        \]
        è omeomorfismo sulla sua immagine indicata con $e_\alpha^n$.
        E vale \[
            X = \bigcup_{\alpha} \Phi_\alpha^n(D_\alpha^n)
        \]
        \item \[
        \Phi_\alpha^n(\partial D^n) \subset X_{n-1},
        \]
        dove \[
            X_n := \bigcup_{k \leq n \\ \alpha \in S_k} e_\alpha^k.
        \]
        \item \[
            C \subset X \text{ chiuso } \iff \Phi_\alpha^{-1}(C) \subset D_\alpha^n \text{ chiuso } \forall n \forall \alpha,
        \]
        che è equivalente a chiedere che la topologia su $X$ è la meno fine a rendere continue tutte le $\Phi_\alpha^n$.
    \end{enumerate}
\end{theorem}

\begin{proposition}
    Per ogni $X_n$ scheletro di un CW-complesso $X$ esiste un intorno $U$ di $X_n$ in $X$ tale che $U$ si retragga per deformazione su $X_n$.
\end{proposition}

\begin{definition} [Complesso di catene cellulare] \nl
    Sia $X$ un CW-complesso con scheletri $\set{X_n}_{n \geq 0}$. \nl
    Si può considerare la seguente composizione di funzioni continue:
    \[
        d_q: H_q(X_q, X_{q-1}) \xrightarrow{\partial_q} H_{q-1}(X_{q-1}) \xhookrightarrow{i_*}H_{q-1}(X_{q-1}, X_{q-2}),
    \]
    ovvero $d_q = i \circ \partial_q$. \nl
    Si può dimostrare che $d_q \circ d_{q+1} = 0$ e dunque: \nl
     posto $C_q := H_q(X_q, X_{q-1})$, si ottiene un complesso di catene $(C_*, d_*)$ detto \textbf{complesso delle catene cellulari} di $X$.
\end{definition}

\begin{proposition} \nl
    Indicato con $e_\alpha^q$ le q-celle di $X$ e $\interior{e_q^\alpha} := \bar{e_\alpha^q} \setminus \mathrm{Int}(e_\alpha^q) = \Phi_\alpha^q(\partial D^q_\alpha)$, vale:
    La mappa:
    \[
        \Phi_\alpha: (D^q_\alpha, \partial D^q_\alpha) \to (\bar{e^q_\alpha}, {e^q_\alpha})
    \]
    induce un isomorfismo in omologia:
    \[
        \Phi_{\alpha *}: H_q(D^q_\alpha, \partial D^q_\alpha) \xrightarrow{\ \cong\ } H_q(\bar{e^q_\alpha}, \interior{e^q_\alpha})
    \]
\end{proposition}

\begin{proof}
    Per $q=0$ non c'è nulla da dimostrare(un punto è sempre omemorfo ad un altro punto). \nl
    Per $q > 0$ si ha il seguente diagramma:
    % https://q.uiver.app/#q=WzAsNixbMCwwLCJIX3EoRF5xLCBcXHBhcnRpYWwgRF5xKSJdLFswLDEsIkhfcShEXnEsIERecSBcXHNldG1pbnVzIFxcezBcXH0pIl0sWzAsMiwiSF9xKFxcbWF0aHJte0ludH0oe0RecX0pLCBcXG1hdGhybXtJbnR9KHtEXnF9KSBcXHNldG1pbnVzIFxcezBcXH0pIl0sWzEsMCwiIEhfcShcXGJhcntlXnFfXFxhbHBoYX0sIFxcaW50ZXJpb3J7ZV5xX1xcYWxwaGF9KSJdLFsxLDEsIiBIX3EoXFxiYXJ7ZV5xX1xcYWxwaGF9LCBcXGJhcntlXnFfXFxhbHBoYX0gXFxzZXRtaW51cyBwKSJdLFsxLDIsIiBIX3EoZV5xX1xcYWxwaGEsIGVecV9cXGFscGhhIFxcc2V0bWludXMgcCkiXSxbMCwxLCJcXGNvbmciLDJdLFsxLDIsIlxcY29uZyIsMl0sWzIsNSwiXFxjb25nIiwyXSxbNCw1LCJcXGNvbmciXSxbMyw0LCJcXGNvbmciXSxbMCwzLCIoXFxQaGlfXFxhbHBoYSlfKiJdXQ==
\[\begin{tikzcd}
	{H_q(D^q, \partial D^q)} & { H_q(\bar{e^q_\alpha}, \interior{e^q_\alpha})} \\
	{H_q(D^q, D^q \setminus \{0\})} & { H_q(\bar{e^q_\alpha}, \bar{e^q_\alpha} \setminus p)} \\
	{H_q(\mathrm{Int}({D^q}), \mathrm{Int}({D^q}) \setminus \{0\})} & { H_q(e^q_\alpha, e^q_\alpha \setminus p)}
	\arrow["{(\Phi_\alpha)_*}", from=1-1, to=1-2]
	\arrow["\cong"', from=1-1, to=2-1]
	\arrow["\cong", from=1-2, to=2-2]
	\arrow["\cong"', from=2-1, to=3-1]
	\arrow["\cong", from=2-2, to=3-2]
	\arrow["\cong"', from=3-1, to=3-2]
\end{tikzcd}\]
    Le mappe sopra sono omeomorfismi perché vengono da retrazioni per deformazione, mentre quelle laterali sotto per il teorema di escissione.
    La mappa in basso è un isomorfismo perché viene dall'omeomorfismo dato dalla mappa caratteristica $\Phi_\alpha$ ristretto a $\mathrm{Int}(D^q)$.
    Quindi per commutatività del diagramma anche la mappa in alto è un isomorfismo.
\end{proof}

\begin{proposition}
    \[
        H_q(X, X_{n-1}) \cong \bigoplus_{\alpha \in S_n} H_q(D^n, \partial D^n) \cong \begin{cases}
            \mathbb{Z}^n & q = n \\
            0 & \text{altrimenti}
        \end{cases}
    \]
\end{proposition}

\begin{remark}
    La mappa $d_q: H_q(X_q, X_{q-1}) \to H_{q-1}(X_{q}, X_{q-1})$ è la mappa di una successione della tripla.
\end{remark}

\begin{lemma}
    \[
        H_q(X_p) = 0 \quad \text{se } q > p,
    \]
    ovvero l'omologia di uno scheletro è nulla nelle dimensioni maggiori della sua.
\end{lemma}

\begin{proof}
    Si fa
\end{proof}

\begin{lemma} \nl
    L'inclusione $i: X_p \hookrightarrow X$ induce un isomorfismo in omologia per ogni $q \leq p$:
    \[
        i_*: H_q(X_p) \xrightarrow{\ \cong\ } H_q(X)
    \]
\end{lemma}

\begin{theorem} [Omologia cellulare] \nl
    Sia $X$ un CW-complesso, e $H_q(C_*, d_*)$ l'omologia nel senso algebrico del complesso delle catene cellulari di $X$. \nl
    Allora 
    \[
        H_q(C_*, d_*) \cong H_q(X) \quad \forall q \in \Z.
    \]
\end{theorem}

\begin{proof}
    Per ogni $n$ si considera il seguente diagramma:

    % https://q.uiver.app/#q=WzAsMTEsWzEsMywiSF97bisxfShYX3tuKzF9LFhfbikiXSxbMywzLCJIX3tufShYX3tufSxYX3tuLTF9KSJdLFs1LDMsIkhfe24tMX0oWF97bi0xfSxYX3tuLTJ9KSJdLFsyLDIsIkhfbihYX24pIl0sWzMsMSwiSF9uKFhfe24rMX0pIl0sWzQsNCwiSF97bi0xfShYX3tuLTF9KSJdLFsxLDEsIjAiXSxbNCwwLCIwIFxcY29uZyBIX3tufShYX3tuKzF9LFhfbikiXSxbNSw1LCIwIFxcY29uZyBIX3tuLTF9KFhfe259KSJdLFswLDMsIlxcY2RvdHMiXSxbNiwzLCJcXGNkb3RzIl0sWzksMF0sWzAsMSwiZF97bisxfSJdLFsxLDIsImRfe259Il0sWzIsMTBdLFswLDMsIlxccGFydGlhbF97bisxfSJdLFszLDEsImpfbiJdLFszLDRdLFs0LDddLFs2LDNdLFsxLDUsIlxccGFydGlhbF97bn0iXSxbNSwyLCJqX3tuLTF9Il0sWzUsOF1d
    \[\begin{tikzcd} [row sep=small, column sep=small, nodes={scale=0.85, transform shape, font=\small}]
        &&&& {0 \cong H_{n}(X_{n+1},X_n)} \\
        & 0 && {H_n(X_{n+1}) \cong H_n(X)} \\
        && {H_n(X_n)} \\
        \cdots & {H_{n+1}(X_{n+1},X_n)} && {H_{n}(X_{n},X_{n-1})} && {H_{n-1}(X_{n-1},X_{n-2})} & \cdots \\
        &&&& {H_{n-1}(X_{n-1})} \\
        &&&&& {0 \cong H_{n-1}(X_{n})}
        \arrow[from=2-2, to=3-3]
        \arrow[from=2-4, to=1-5]
        \arrow[from=3-3, to=2-4]
        \arrow["{j_n}", from=3-3, to=4-4]
        \arrow[from=4-1, to=4-2]
        \arrow["{\partial_{n+1}}", from=4-2, to=3-3]
        \arrow["{d_{n+1}}", from=4-2, to=4-4]
        \arrow["{d_{n}}", from=4-4, to=4-6]
        \arrow["{\partial_{n}}", from=4-4, to=5-5]
        \arrow[from=4-6, to=4-7]
        \arrow["{j_{n-1}}", from=5-5, to=4-6]
        \arrow[from=5-5, to=6-6]
    \end{tikzcd}\]

    \begin{enumerate}
        \item Poiché $H_n(X_{n+1}, X_n) = 0$, vale che $H_n(X_{n+1}) \cong \frac{H_n(X_n)}{\mathrm{Im(\partial_{n+1})}}$. 
        \item Per il lemma precedente vale che $H_n(X_{n+1}) \cong H_n(X)$, dunque $H_n(X) \cong \frac{H_n(X_n)}{\mathrm{Im(\partial_{n+1})}}$.
        \item $j_n$ è iniettiva dunque $H_n(X_n) \cong \mathrm{Im}(j_n)$.
        \item Poiché $j_n$ è iniettiva e $d_{n+1} = j_n \circ \partial_{n+1}$, si ha che $\mathrm{Im}(d_{n+1}) \cong \mathrm{Im}(\partial_{n+1})$ e $\mathrm{Ker}(d_{n+1}) \cong \mathrm{Ker}(\partial_{n+1})$.
        \item Dunque $H_n(X_n) \cong \mathrm{Im}(j_n) \cong \mathrm{Ker}(\partial_n) \cong \mathrm{Ker}(d_n)$.
        \item Infine $H_n(X) \cong \frac{H_n(X_n)}{\mathrm{Im}(\partial_{n+1})} \cong \frac{\mathrm{Ker}(d_n)}{\mathrm{Im}(d_{n+1})} = H_n(C_*, d_*)$.
    \end{enumerate}
\end{proof}

\begin{example} \nl
    \begin{enumerate}
        \item Calcolo dell'omologia di $X = \C \mathbb{P}^n$. \nl
        Grazie alle carte affini è possibile dare una struttura di CW-complesso a $\C \mathbb{P}^n$ con una cella in ogni dimensione pari fino a $2n$:
        \[
            \C \mathbb{P}^n = \C^n \cup \C \mathbb{P}^{n-1} = \cdots = \C^n \cup \cdots \cup \C = \bigcup_{k=0}^n e^{2k}
        \]
        Dunque il complesso delle catene cellulare è:
        \[
            0 \to \mathbb{Z} \xrightarrow{0} 0 \xrightarrow{0} \mathbb{Z} \xrightarrow{0} \cdots \xrightarrow{0} \mathbb{Z} \to 0
        \]
        e dunque l'omologia è:
        \[
            H_q(\C \mathbb{P}^n) = \begin{cases}
                \mathbb{Z} & q \text{ pari }, 0 \leq q \leq 2n \\
                0 & \text{ altrimenti }
            \end{cases}
        \]
        \item Calcolo dell'omologia di $X = \R \mathbb{P}^n$. \nl
        Anche in questo caso le carte affini inducono una struttura di CW-complesso con una cella in ogni dimensione fino a $n$:
        \[
            \R \mathbb{P}^n = \R^n \cup \R \mathbb{P}^{n-1} = \cdots = \R^n \cup \cdots \cup \R = \bigcup_{k=0}^n e^{k}.
        \]
        Tuttavia c'è bisogno di teoria per il calcolo delle mappe di bordo del complesso cellulare.
        Assumendo risultati che vedremo in seguito, si ha che il complesso delle catene cellulare è:
        \[
            0 \to H_q(X) = \mathbb{Z} \xrightarrow{1 + (-1)^{q}} H_q(X) = \mathbb{Z} \xrightarrow{1 + (-1)^{q-1}} \mathbb{Z} \xrightarrow{2} H_0(X) = \mathbb{Z} \xrightarrow{0} \tilde{H}_0(X) = \mathbb{Z} \xrightarrow{} 0
        \]
        Dunque l'omologia è:
        \[
            H_q(\R \mathbb{P}^n) = \begin{cases}
                \mathbb{Z} & q = 0 \\
                \mathbb{Z}_2 & q \text{ dispari }, 1 \leq q < n \\
                \mathbb{Z} & q = n \text{ e } n \text{ dispari } \\
                0 & \text{ altrimenti }
            \end{cases}
        \]
    \end{enumerate}
\end{example}

\begin{proposition} \nl
    Dato $X$ un CW-complesso, la coppia $(X, X_{n-1})$ è una buona coppia. \nl1
\end{proposition}

\begin{proposition} [Formula per la mappa di bordo cellulare] \nl
    Sia $X$ un CW-complesso e $d_n: H_n(X_n, X_{n-1}) \to H_{n-1}(X_{n-1}, X_{n-2})$ la mappa di bordo cellulare. \nl
    Allora per ogni $\alpha \in S_n$ si ha:
    \[
        d_n(e_\alpha^n) = \sum_{\beta \in S_{n-1}} \deg(\Delta_{\alpha \beta}^n) e_\beta^{n-1}
    \]
    dove $\Delta_{\alpha \beta}^n: S^{n-1} \to S^{n-1}$ è la composizione:
    \[
        S^{n-1}_\alpha \xrightarrow{\varphi_\alpha^n} X_{n-1} \xrightarrow{q_\beta} X_{n-1}/(X_{n-1} \setminus e_\beta^{n-1}) \cong S^{n-1}_\beta
    \]
    con $q_\beta$ la proiezione canonica.
\end{proposition}

\begin{proof}
    Si consideri il seguente diagramma commutativo:
    % https://q.uiver.app/#q=WzAsOCxbMCwxLCJIX24oWF97bn0sIFhfe24tMX0pIl0sWzEsMCwiSF97bi0xfShYX3tuLTF9LCBYX3tuLTJ9KSJdLFsxLDEsIkhfe24tMX0oWF97bi0xfSkiXSxbMiwwLCJIX3tuLTF9KFhfe24tMX0vWF97bi0yfSwgW1hfe24tMn1dKSJdLFsyLDEsIlxcdGlsZGV7SH1fe24tMX0oWF97bi0xfS9YX3tuLTJ9KSJdLFsyLDIsIlxcdGlsZGV7SH1fe24tMX0oU157bi0xfV9cXGJldGEpIl0sWzEsMiwiSF97bi0xfShTXntuLTF9X1xcYWxwaGEpIl0sWzAsMiwiSF97bn0oRF5uX1xcYWxwaGEsIFNee24tMX1fXFxhbHBoYSkiXSxbMSwzLCJcXGNvbmciXSxbMyw0LCJcXGNvbmciXSxbMiw0LCJcXHBpIl0sWzIsMSwial8qIiwyXSxbMCwxLCJkX3EiXSxbMCwyLCJcXHBhcnRpYWxfcSJdLFs3LDAsIihcXFBoaV9cXGFscGhhKV8qIl0sWzYsMiwiKFxcdmFycGhpX1xcYWxwaGEpXyoiXSxbNCw1LCJcXHBpX1xcYmV0YSJdLFs3LDZdLFs2LDUsIlxcRGVsdGFfe1xcYWxwaGFcXGJldGF9Il1d
\[\begin{tikzcd}
	& {H_{n-1}(X_{n-1}, X_{n-2})} & {H_{n-1}(X_{n-1}/X_{n-2}, [X_{n-2}])} \\
	{H_n(X_{n}, X_{n-1})} & {H_{n-1}(X_{n-1})} & {\tilde{H}_{n-1}(X_{n-1}/X_{n-2})} \\
	{H_{n}(D^n_\alpha, S^{n-1}_\alpha)} & {H_{n-1}(S^{n-1}_\alpha)} & {\tilde{H}_{n-1}(S^{n-1}_\beta)}
	\arrow["\cong", from=1-2, to=1-3]
	\arrow["\cong", from=1-3, to=2-3]
	\arrow["{d_q}", from=2-1, to=1-2]
	\arrow["{\partial_q}", from=2-1, to=2-2]
	\arrow["{j_*}"', from=2-2, to=1-2]
	\arrow["\pi", from=2-2, to=2-3]
	\arrow["{\pi_\beta}", from=2-3, to=3-3]
	\arrow["{(\Phi_\alpha)_*}", from=3-1, to=2-1]
	\arrow[from=3-1, to=3-2]
	\arrow["{(\varphi_\alpha)_*}", from=3-2, to=2-2]
	\arrow["{\Delta_{\alpha\beta}}", from=3-2, to=3-3]
\end{tikzcd}\]
\end{proof}

\begin{definition} [Caratteristica di Eulero]
    Per il teorema di struttura dei gruppi abeliani finitamente generati, ogni gruppo abeliano finitamente generato ha una parte libera ed una parte di torsione. \nl
    Se si definisce il rango di un gruppo abeliano finitamente generato come il rango della sua parte libera, si possono definire i
    numeri di Betti di uno spazio $X$ come \[
        b_q(X) := \mathrm{rank}(H_q(X)).
    \]
    La \textbf{caratteristica di Eulero} di $X$ è definita come:
    \[
        \chi(X) := \sum_{q=0}^{\infty} (-1)^q b_q(X).
    \] 

\end{definition}

\begin{definition}
    La caratteristica di Eulero si pùo definire analogamente per un complesso di catene $(C_*, d_*)$ come:
    \[
        \chi(C_*, d_*) := \sum_{q=0}^{\infty} (-1)^q \mathrm{rank}(H_q(C_*, d_*)).
    \]
\end{definition}

\begin{proposition}
    Un complesso di catene $(C_*, d_*)$ aciclico ha caratteristica di Eulero nulla.
\end{proposition}


\section{Omologia e Coomologia a coefficienti in un modulo}

\subsection{}
\begin{definition} [Prodotto tensoriale di gruppi abeliani] \nl
    Siano $A, B$ due gruppi abeliani, si indica con $F(A, B)$ il gruppo abeliano libero generato da tutte le coppie $(a, b)$ con $a \in A$ e $b \in B$. \nl
    Si definisce il \textbf{prodotto tensoriale} di $A$ e $B$ come il quoziente:
    \[
        A \otimes B := F(A, B) / R
    \]
    dove $R$ è il sottogruppo generato dagli elementi:
    \begin{enumerate}
        \item $(a_1 + a_2, b) - (a_1, b) - (a_2, b)$ per ogni $a_1, a_2 \in A$ e $b \in B$,
        \item $(a, b_1 + b_2) - (a, b_1) - (a, b_2)$ per ogni $a \in A$ e $b_1, b_2 \in B$,
    \end{enumerate}
\end{definition}

\begin{theorem} [Proprietà universale del prodotto tensoriale] \nl
    Dati $A, B$ gruppi abeliani, $A \times B$ prodotto diretto e $A \otimes B$ prodotto tensoriale, vale che: \nl
    Per ogni gruppo abeliano $G$ e ogni applicazione bilineare $\varphi: A \times B \to G$ esiste un'unica applicazione di gruppi abeliani $\tilde{\varphi}: A \otimes B \to G$ tale che il seguente diagramma commuta:
    \[
    \begin{tikzcd}
        A \times B \arrow[r, "\varphi"] \arrow[d, "\otimes"] & G \\
        A \otimes B \arrow[ru, "\tilde{\varphi}"']
    \end{tikzcd}
    \]
\end{theorem}

\begin{proof}
    Sia
    \[
        \Phi: F(A, B) \to G: (a, b) \mapsto \varphi(a, b).
    \]
    che esiste per proprietà universale dei gruppi liberi. \nl
    Poiché $\varphi$ è bilineare, si ha che $R \subset \ker(\Phi)$, dunque esiste un'unica applicazione $\tilde{\varphi}: A \otimes B \to G$ tale che $\tilde{\varphi} \circ \otimes = \Phi$ per il primo teorema di isomorfismo.
\end{proof}

\begin{remark}
    Per ogni gruppo abeliano $G$ si ha che \[
        \mathbb{Z} \otimes G \cong G,
    \]
    dove l'isomorfismo è dato da $n \otimes g \mapsto ng$.
\end{remark}

\begin{proposition}
    Data una successione esatta corta di gruppi abeliani:
    \[
        0 \to A \xrightarrow{i} A' \xrightarrow{\pi} A'' \to 0,
    \]
    si ha una successione di gruppi abeliani corta esatta a destra:
    \[
        A \otimes G \xrightarrow{i \otimes \mathrm{id}} A' \otimes G \xrightarrow{\pi \otimes \mathrm{id}} A'' \otimes G \to 0.
    \]
    che in generale non è esatta a sinistra.
\end{proposition}

\begin{example} 
    In generale la successione non è esatta a sinistra, infatti data la successione esatta corta:
    \[
        0 \to \mathbb{Z} \xrightarrow{2} \mathbb{Z} \xrightarrow{\pi} \mathbb{Z} / \mathbb{Z}_2 \to 0,
    \]
    si ha la successione:
    \[
        \mathbb{Z} \times \mathbb{Z}_2 \xrightarrow{2 \otimes \mathrm{id}} \mathbb{Z} \times \mathbb{Z}_2 \xrightarrow{\pi \otimes \mathrm{id}} (\mathbb{Z} / \mathbb{Z}_2) \times \mathbb{Z}_2 \to 0,
    \]
    ma vale che $2 \otimes 1 = 1 \otimes 2 = 1 \otimes 0 = 0$ e dunque la mappa a sinsitra è nulla e di conseguenza non iniettiva.
\end{example}

\begin{proof}
    La mappa $\pi \otimes \mathrm{id}$ è suriettiva perché per ogni $(a'', g) = (\pi \otimes \mathrm{id})(a', g)\in A'' \otimes G$ dove $a' \in A'$ è dato dalla surgettività di $\pi$. \nl
\end{proof}

\begin{example}
    \begin{enumerate}
        \item $
            A \otimes B \cong B \otimes A
        $
        \item $
            (A \otimes B) \otimes C \cong A \otimes (B \otimes C)
        $
        \item Se $A,B$ liberi con base rispettivamente $\set{a_i}_{i \in I}$ e $\set{b_j}_{j \in J}$, allora $A \otimes B$ è libero con base $\set{a_i \otimes b_j}_{i \in I, j \in J}$.
    \end{enumerate}
\end{example}

\begin{definition}
    Dato un complesso $(C_*, \partial_*)$ di moduli liberi e un gruppo abeliano libero $G$, si definisce il \textbf{complesso tensore} come il complesso $(C_* \otimes G, \partial_* \otimes \mathrm{id})$ dove:
    \[
        (C_* \otimes G)_q := C_q \otimes G,
    \]
    e la sua omologia
    \[
        H_q(C_* \times G) := \frac{\ker(\partial_q \otimes 1)}{\Img(\partial_{q-1} \otimes 1)}
    \]
\end{definition}

\begin{remark}
    Un morfismo di complessi $\varphi: C_* \to D_*$ induce un morfismo di complessi $\varphi \otimes \id: C_* \otimes G \to C_* \otimes G$.
\end{remark}

\begin{remark}
    Data una successione esatta corta di complessi di moduli \textbf{liberi}:
    \[
        0 \to A_* \xrightarrow{f_*} B_* \xrightarrow{g_*} C_* \to 0,
    \]
    si ha una successione esatta corta di complessi:
    \[
        0 \to A_* \otimes G \xrightarrow{f_* \otimes \mathrm{id}} B_* \otimes G \xrightarrow{g_* \otimes \mathrm{id}} C_* \otimes G \to 0.
    \]
    Attenzione: Se i moduli non sono liberi la successione potrebbe non essere esatta a sinistra.
\end{remark}

\begin{definition} [Omologia a coefficienti in un gruppo abeliano] \nl
    Sia $X$ uno spazio topologico e $G$ un gruppo abeliano, si definisce l'\textbf{omologia di $X$ con coefficienti in $G$} come:
    \[
        H_q(X; G) := H_q(C_*(X) \otimes G),
    \]
    dove $C_*(X)$ è il complesso delle catene singolari di $X$.
\end{definition}

\begin{definition}[Funtore]
    
\end{definition}
\end{document}