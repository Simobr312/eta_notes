\documentclass[]{article}

% Packages
\usepackage[utf8]{inputenc} % For UTF-8 encoding
\usepackage[T1]{fontenc}    % For proper Italian accents
\usepackage[italian]{babel} % Italian language support
\usepackage{amsmath, amsfonts, amssymb} % Math packages
\usepackage{graphicx}       % For including images
\usepackage{hyperref}       % For hyperlinks
\usepackage{bookmark}       % For improved PDF bookmarks and rerunfilecheck warning
\usepackage{tikz}           % For diagrams
\usepackage{geometry}       % For page layout
\usepackage{fancyhdr}       % For custom headers/footers
\usepackage{enumitem}       % For custom lists
\usepackage{xcolor}         % For colored text
\usepackage{mathtools}      % For additional math tools
\usepackage{amsthm}         % For theorem environments
\usepackage{quiver}        % For commutative diagrams

% Load custom commands
\usepackage{commands} 
% Theorem styles

% Page layout
\geometry{a4paper, margin=1in}
\pagestyle{fancy}
\fancyhf{}
\fancyhead[L]{Note del corso di Elementi di Topologia Algebrica}
\fancyhead[R]{\thepage}

\begin{document}

\title{Appunti di Elementi di Topologia Algebrica}
\author{Simone Riccio \\ \small{(dalle lezioni del Prof. Mario Salvetti)}}
\date{\today}

\maketitle

\tableofcontents

\section{Omologia Singolare}

\subsection{Omologia ridotta}
\begin{definition} [Complesso di catene aumentato] \nl
 Sia $X$ spazio topologico. Il \textbf{complesso di catene aumentato} $\tilde{C}(X)$ di $X$ è il complesso di catene:
 \[
    \cdots \xrightarrow{\partial_{n+1}} \tilde{C}_n(X) \xrightarrow{\partial_n} \tilde{C}_{n-1}(X) \xrightarrow{\partial_{n-1}} \cdots \xrightarrow{\partial_1} \tilde{C}_0(X) \xrightarrow{\epsilon} \mathbb{Z} \to 0
 \]
 dove $\epsilon: \tilde{C}_0(X) \to \mathbb{Z}$ è l'omomorfismo di aumentazione definito da:
 \[
    \epsilon\left(\sum_i n_i \sigma_i\right) = \sum_i n_i
 \]
 per ogni catena singolare $\sum_i n_i \sigma_i \in \tilde{C}_0(X)$.
\end{definition}

\begin{proposition}
 Sia $X$ spazio topologico, $H(X), \tilde{H(X)}$ rispettivamente l'omologia singolare e l'omologia singolare ridotta con coefficienti in $\mathbb{Z}$. \nl
 Vale:
 \[
    H_n(X) \cong \begin{cases}
        \tilde{H}_n(X) & n > 0 \\
        \tilde{H}_0(X) \oplus \mathbb{Z} & n = 0
    \end{cases}
 \]
\end{proposition}


\begin{proof}
    Il fatto di dimostra, applicando il teorema del morfismo di connessione alla successione esatta:
    \[
          0 \to \tilde{C}(X) \xrightarrow{} C(X) \xrightarrow{\epsilon} \mathbb{Z} \to 0
    \]
    Qui intendo con $\mathbb{Z}$ il complesso di catene che in grado 0 è $\mathbb{Z}$ e in tutti gli altri gradi è $\set{0}$. \nl
    Dunque si ha in omologia la successione esatta lunga:
    \[
        \cdots \to H_1(\mathbb{Z}) \cong \set{0} \to \tilde{H}_0(X) \to H_0(X) \to H_0(\mathbb{Z}) \cong \mathbb{Z} \to 0 \to \cdots
    \]
    e dato che la successione \textbf{scinde}, si ottiene l'isomorfismo voluto.
\end{proof}

\begin{remark} [Cosa significa che la successione scinde?] \nl
    Vedi teorema 102 del libro di Algebra 2. \nl
    Una successione esatta corta:
    \[
        0 \to A \xrightarrow{f} B \xrightarrow{g} C \to 0
    \]
    si dice che \textbf{scinde} se esiste un isomorfismo $\varphi: B \to A \oplus C$ tale che il seguente diagramma commuti:
    \[
        \begin{tikzcd}
            0 \arrow{r} & A \arrow{r}{f} \arrow[swap]{d}{\mathrm{id}_A} & B \arrow{r}{g} \arrow{d}{\varphi} & C \arrow{r} \arrow[swap]{d}{\mathrm{id}_C} & 0 \\
            0 \arrow{r} & A \arrow{r} & A \oplus C \arrow{r} & C \arrow{r} & 0
        \end{tikzcd}
    \]
    Ed è equivalente a dire che esiste un morfismo $s: C \to B$ tale che $g \circ s = \mathrm{id}_C$ (sezione destra) oppure un morfismo $r: B \to A$ tale che $r \circ f = \mathrm{id}_A$ (sezione sinistra).
    
\end{remark}

\subsection{Omotopia di catene e di mappe continue}
\begin{definition} [Omotopia di complessi di catene] \nl
        Siano $(C_*, \partial_*^C)$ e $(D_*, \partial_*^D)$ due complessi di catene e $f_\#, g_\#: C_* \to D_*$ due mappe di complessi di catene. \nl
        Un'\textbf{omotopia} tra $f_\#$ e $g_\#$ è una famiglia di omomorfismi di gruppi abeliani:
        \[
            p_q: C_q \to D_{q+1}
        \]
        tale che:
        \[
            \partial_{q}^D \circ p_q + p_{q-1} \circ \partial_q^C = g_q - f_q
        \]
\end{definition}

\begin{proposition} [Mappe di catene omotope di inducono la stessa mappa in omologia] \nl
    Siano $(C_*, \partial_*^C)$ e $(D_*, \partial_*^D)$ due complessi di catene e $f_\#, g_\#: C_* \to D_*$ due mappe di complessi di catene omotope. \nl
    Allora le mappe indotte $f_*, g_*: H_n(C_*) \to H_n(D_*)$ sono uguali per ogni $n \geq 0$.
\end{proposition}

\begin{proof} \nl
    Sia $[c] \in H_n(C_*)$ una classe di omologia, con $c \in Z_n(C_*)$. \nl
    Allora:
    \[
        g_n(c) - f_n(c) = \partial_n^D (p_n(c)) + p_{n-1}(\underbrace{\partial_n^C(c)}_{=0}) = \partial_n^D (p_n(c))
    \]
    Dunque $g_n(c)$ e $f_n(c)$ differiscono per un bordo, e quindi rappresentano la stessa classe di omologia in $H_n(D_*)$. \nl
    Quindi $g_*([c]) = f_*([c])$ per ogni $[c] \in H_n(C_*)$, da cui si conclude che $g_* = f_*$.
\end{proof}
\begin{definition} [Omotopia di mappe continue] \nl
    Siano $X, Y$ spazi topologici e $f, g: X \to Y$ due mappe continue. \nl
    Una \textbf{omotopia} tra $f$ e $g$ è una mappa continua:
    \[
        F: X \times I \to Y
    \]
    tale che:
    \[
        F(x, 0) = f(x), \quad F(x, 1) = g(x) \quad \forall x \in X
    \]
\end{definition}

\begin{theorem} [Invarianza omotopica dell'omologia singolare] \nl
    Siano $X, Y$ spazi topologici e $f, g: X \to Y$ due mappe continue omotope. \nl
    Allora le mappe indotte $f_*, g_*: H_n(X) \to H_n(Y)$ sono uguali per ogni $n \geq 0$. \nl
    Di conseguenza:
        \[X \sim Y \implies H_n(X) \cong H_n(Y) \text{ per ogni } n \geq 0.\]
\end{theorem}

\begin{proof} \nl
    \begin{enumerate}
        \item Se $F$ è un'omotopia tra $f$ e $g$, definiamo le mappe:
            \[
                \eta_s: X \to X \times I: x \mapsto (x, s) \quad s \in I
            \]
            che formano un'omotopia tra $\eta_0$ e $\eta_1$. \nl
            Si ha che $f = F \circ \eta_0$ e $g = F \circ \eta_1$. \nl
            Dunque se mostriamo che $(\eta_0)_* = (\eta_1)_*$, si ottiene per funtorialità che $f_* = g_*$.
        \item  Assumendo che due mappe omotope nel senso dei complessi di catene inducono la stessa mappa in omologia,
            dobbiamo mostrare che $(\eta_0)_\#$ e $(\eta_1)_\#*$ sono omotope nel senso dei complessi di catene. \nl
        Vogliamo costruire un'omotopia di complessi di catene:
            % https://q.uiver.app/#q=WzAsMTIsWzEsMCwiQ197cSsxfShYKSJdLFsyLDAsIkNfe3F9KFgpIl0sWzMsMCwiQ197cS0xfShYKSJdLFsxLDEsIkNfe3ErMX0oWSkiXSxbMiwxLCJDX3txfShZKSJdLFszLDEsIkNfe3EtMX0oWSkiXSxbNCwxXSxbNCwwXSxbMCwwXSxbMCwxXSxbMiw0LCJcXGJ1bGxldCJdLFsyLDMsIlxcYnVsbGV0Il0sWzAsMV0sWzEsMl0sWzMsNF0sWzUsNl0sWzQsNV0sWzIsN10sWzgsMF0sWzksM10sWzAsMywiXFxldGFfezAqfSIsMl0sWzAsMywiXFxldGFfezEqfSIsMCx7Im9mZnNldCI6LTN9XSxbMSw0LCJcXGV0YV97MCp9IiwyXSxbMSw0LCJcXGV0YV97MSp9IiwwLHsib2Zmc2V0IjotM31dLFsyLDUsIlxcZXRhX3swKn0iLDJdLFsyLDUsIlxcZXRhX3sxKn0iLDAseyJvZmZzZXQiOi0zfV0sWzEwLDExXV0=
            \[\begin{tikzcd}
                {} & {C_{q+1}(X)} & {C_{q}(X)} & {C_{q-1}(X)} & {} \\
                {} & {C_{q+1}(Y)} & {C_{q}(Y)} & {C_{q-1}(Y)} & {} \\
                \\
                \arrow[from=1-1, to=1-2]
                \arrow[from=1-2, to=1-3]
                \arrow["{\eta_{0\#}}"', from=1-2, to=2-2]
                \arrow["{\eta_{1\#}}", shift left=3, from=1-2, to=2-2]
                \arrow[from=1-3, to=1-4]
                \arrow["{\eta_{0\#}}"', from=1-3, to=2-3]
                \arrow["{\eta_{1\#}}", shift left=3, from=1-3, to=2-3]
                \arrow[from=1-4, to=1-5]
                \arrow["{\eta_{0\#}}"', from=1-4, to=2-4]
                \arrow["{\eta_{1\#}}", shift left=3, from=1-4, to=2-4]
                \arrow[from=2-1, to=2-2]
                \arrow[from=2-2, to=2-3]
                \arrow[from=2-3, to=2-4]
                \arrow[from=2-4, to=2-5]
            \end{tikzcd}\]
        Cioè una mappa $p: C_q(X) \to C_{q+1}(Y)$ che chiameremo \textbf{operatore prisma} tale che:
        \[
            \partial_{q-1}^Y \circ p_q + p_{q-1} \circ \partial_q^X = \eta_{1\#} - \eta_{0\#}
        \]
        \item Definiamo l'operatore prisma $p_q: C_q(X) \to C_{q+1}(Y)$ in maniera funtoriale come segue. \nl
        Se $p_q$ è funtoriale allora per ogni $f: X \to Y$ il seguente diagramma commuta:
        \[
        \begin{tikzcd}
            C_q(X) \arrow{r}{p_q} \arrow{d}{f_*} & C_{q+1}(X \times I) \arrow{d}{(f \times \mathrm{id}_I)_*} \\
            C_q(Y) \arrow{r}{p_q} & C_{q+1}(Y \times I)
        \end{tikzcd}
        \]
        Sia ora $\sigma: \Delta^q \to X$ un q-simplesso singolare di $X$. \nl
        Vale che $\sigma = \sigma_* (\mathrm{id}_{\Delta^q})$ \nl
        Di conseguenza per funtorialità vale:
        \[
        p_q(\sigma) = (\sigma \times \mathrm{id}_I)_* (p_q(\mathrm{id}_{\Delta^q}))
        \]
        Per cui basta definire $p_q(\mathrm{id}_{\Delta^q})$ per definire l'operatore prisma in generale. \nl
        \item Definiamo dunque $p_q(\mathrm{id}_{\Delta^q}): \Delta^{q+1} \to \Delta^q \times I$.
        Suddividiamo $\Delta^{q} \times I$ in $q+1$ simplessi $[v_0 \cdots v_i w_i \cdots w_q]$ per $i = 0, \ldots, q$, dove:
        \[
            \Delta^{q} \times \set{0} = [v_0 \cdots v_q], \quad \Delta^{q} \times \set{1} = [w_0 \cdots w_q]
        \]
        e l'operatore prisma è definito come:
        \[
            p_q(\mathrm{id}_{\Delta^q}) := \sum_{i=0}^q (-1)^i [v_0 \cdots v_i w_i \cdots w_q]
        \]
        Bisogna verificare che questa definizione soddisfi la proprietà richiesta:
        \[
            \partial_{q-1}^{X \times I} \circ p_q + p_{q-1} \circ \partial_q^X = \eta_{1\#} - \eta_{0\#}
        \]
        Si verifica calcolando i due termini a sinistra e sommando.
    \end{enumerate}
\end{proof}

\subsection{Successione di Mayer-Vietoris}

\begin{definition} [Catene $\mathcal{U}$-piccole]
    Sia $\mathcal{U}$ un ricoprimento aperto di uno spazio topologico $X$. \nl
    Una catena $c = \sum_{\sigma} \nu_\sigma \sigma \in C_q(X)$ si dice \textbf{$\mathcal{U}$-piccola} se
    \[
        \forall \sigma \text{ con } \nu_\sigma \neq 0, \exists U \in \mathcal{U} : \mathrm{Im}(\sigma) \subseteq U.
    \]
    Si noti che se $c$ è $\mathcal{U}$-piccola, allora anche il suo bordo $\partial c$ è $\mathcal{U}$-piccolo. \nl
    Quindi le catene $\mathcal{U}$-piccole formano un sottocomplesso di catene di $C(X)$, che indichiamo con $C^{\mathcal{U}}(X)$.
\end{definition}

\begin{theorem} [Mayer-Vietoris] \nl
    Sia $X$ spazio topologico, $A, B \subset X$ tali che $X = \mathrm{Int}(A) \cup \mathrm{Int}(B)$. \nl
    Allora la successione esatta corta definita da 
    \[
        0 \to C_*(A \cap B) \xrightarrow{\varphi} C_*(A) \oplus C_*(B) \xrightarrow{\psi} C_*(X) \to 0
    \]
    dove $\varphi(c) = (c, c)$ e $\psi(a, b) = a - b$, induce in omologia la successione esatta lunga:
    \[
        \cdots \to H_n(A \cap B) \xrightarrow{\varphi_*} H_n(A) \oplus H_n(B) \xrightarrow{\psi_*} H_n(X) \xrightarrow{\Delta_n} H_{n-1}(A \cap B) \to \cdots
    \]
    con $\Delta: H(X) \to H(A \cap B)$ di grado $-1$.
\end{theorem}

\begin{proof} \nl
    Consideriamo il ricoprimento aperto $\mathcal{U} = \set{\mathrm{Int}(A), \mathrm{Int}(B)}$ di $X$. \nl
    \begin{enumerate}
        \item Dimostriamo prima assumendo che $H_q(X) \cong H^{\mathcal{U}}_q(X)$
        Vale per ogni $q$ che $C^{\mathcal{U}}_q(X) = C_q(A) + C_q(B)$, dunque la mappa $\psi: C_q(A) \oplus C_q(B) \to C^{\mathcal{U}}_q(X)$ definita da $\psi(c) = (a, b)$ con $c = a - b$, è suriettiva. \nl
        E dunque è esatta la successione:
        \[
            0 \to C_q(A \cap B) \xrightarrow{\varphi} C_q(A) \oplus C_q(B) \xrightarrow{\psi} C^{\mathcal{U}}_q(X) \to 0
        \]
        e per il teorema del morfismo di connessione si ottiene la successione esatta lunga in omologia:
        \[
            \cdots \to H_n(A \cap B) \xrightarrow{\psi_*} H_n(A) \oplus H_n(B) \xrightarrow{\varphi_*} H^{\mathcal{U}}_n(X) \xrightarrow{\Delta_n} H_{n-1}(A \cap B) \to \cdots
        \]
    \end{enumerate}
\end{proof}

\begin{proposition} [Omologia della sfera] \nl
    Sia $S^n$ la sfera n-dimensionale. \nl
    Allora:
    \[
        \tilde{H_q}(S^n) \cong \begin{cases}
            \mathbb{Z} & q = n \\
            0 & \text{altrimenti}
        \end{cases}
    \]
\end{proposition}

\begin{proof}
    Sia $S_n \cong A \cup B$ dove $A := S^n \setminus \{(0, \cdots, 1)\}, B := S^n \setminus \{(0, \cdots, -1)\}$. \nl
    $A, B$ sono contraibili, dunque $\tilde{H_q}(A) \cong \tilde{H_q}(B) \cong 0$ per ogni $q$. \nl
    Mentre $A \cap B$ si retrae per deformazione sull'equatore della sfera e dunque $\tilde{H_q}(A \cap B) \cong \tilde{H_q}(S^{n-1})$. \nl
    Applicando la successione di Mayer-Vietoris si ottiene la successione esatta lunga:
    \[
        \cdots \to \tilde{H_q}(S^{n-1}) \to 0 \to \tilde{H_q}(S^n) \to \tilde{H}_{q-1}(S^{n-1}) \to 0 \to \cdots
    \]

    Da cui si ottiene l'isomorfismo:
    \[
        \tilde{H_q}(S^n) \cong \tilde{H}_{q-1}(S^{n-1})
    \]

    Quindi dato che $S^0 = \set{0, 1}$ e dunque ha due componenti connesse, vale il caso base
    \[
        H_n(S^0) := \begin{cases}
            \mathbb{Z} & n = 0 \\
            0 & n \neq 0
        \end{cases}
    \] 
    da cui segue la tesi per induzione.
\end{proof}

\begin{theorem} [Punto fisso di Brower] \nl
    Sia $f: D^n \to D^n$ continua allora $\exists x \in D^n$ tale che $f(x) = x$ 
\end{theorem}

\begin{proof} \nl
    \begin{enumerate}
        \item Dimostriamo che non esiste una retrazione $r: D^n \to \partial{D^n} \cong S^{n-1}$. \nl
        Supponiamo per assurdo che esista tale retrazione. \nl
        Allora vale che l'identità su $S^{n-1}$ si fattorizza come:
        \[
            S^{n-1} \xhookrightarrow{i} D^n \xrightarrow{r} S^{n-1}
        \]
        dove $i$ è l'inclusione. \nl
        Per funtorialità in omologia si avrebbe che $(\mathrm{id}_{S^{n-1}})_*$ si fattorizza come:
        \[
            \mathbb{Z} \xhookrightarrow{i_*} 0 \xrightarrow{r_*} \mathbb{Z}
        \]
        assurdo.
        \item Supponiamo ora che esista una mappa continua $f: D^n \to D^n$ senza punti fissi. \nl
        Allora possiamo definire una retrazione $r: D^n \to S^{n-1}$ come:
        \[
            r(x) = \text{intersezione tra } S^{n-1} \text{ e la retta che congiunge } f(x) \text{ e } x
        \]
        
        Ma questo contraddice il punto precedente.
    \end{enumerate}
\end{proof}

\begin{proposition} \nl
    Sia $\mathrm{deg}: [S^n, S^n] \to \mathbb{Z}$ il grado di una mappa continua $f: S^n \to S^n$. \nl 
    Se $f: S^n \to S^n$ è senza punti fissi allora $f$ è omotopa alla mappa antipodale e quindi $\mathrm{deg}(f) = (-1)^{n+1}$.
\end{proposition}

\begin{proof}
    Supponiamo che $f$ sia senza punti fissi. \nl
    Allora possiamo definire un'omotopia tra $f$ e la mappa antipodale $a: S^n \to S^n$ come:
    \[
        F: S^n \times I \to S^n
    \]
    (Non mi è chiaro come continuare questa dimostrazione)
    Dunque per invarianza omotopica dell'omologia singolare vale che $\mathrm{deg}(f) = \mathrm{deg}(a) = (-1)^{n+1}$.
\end{proof}

\begin{theorem} [Palla pelosa] \nl
    Sia $v: S^n \to \R^{n+1}$ continua e tale che $v(x)$ sia ortogonale a $x$ e tangente ad $S^n$. \nl
\end{theorem}

\subsection{Relazione tra $H_1(X)$ e $\pi_1(X)$}

Per vedere i disegni di questa parte, che sono molto importanti, guardare le note di Simone Saccani. \nl
\begin{remark}
    Un cammino chiuso $\sigma: (I, \partial{I}) \to (X, x_0)$ può essere visto come $1$-simplesso signolare $\sigma: \Delta^1 \to X$ tale che $\partial{\sigma} = 0$. \nl
    Mostriamo che la classe di $\sigma$ in omologia di $H_1(X)$ dipende solo dalla classe di omotopia di $\sigma$ in $\pi_1(X, x_0)$ e che la giunzione di cammini corrisponde alla somma in omologia. \nl
    \begin{enumerate}
        \item Siano $\sigma \sim \sigma'$ lacci omotopi ad estremi fissi con omotopia $F: I \times I \to X$. \nl
        Dividiamo $I \times I$ in due simplessi: $[e_0 e_1 e_1']$ e $[e_0 e_1' e_0']$ con:
        \[
            e_0 = (0, 0), \quad e_1 = (1, 0), \quad e_1' = (1, 1), \quad e_0' = (0, 1)
        \]
        Consideriamo la catena singolare: $c = F \circ [e_0 e_1 e_1'] - F \circ [e_0 e_1' e_0']$. \nl
        Si verifica che $\partial{c} = \sigma' - \sigma$, dunque $[\sigma] = [\sigma']$ in $H_1(X)$.
        \item Sia $F: I \times I \to X$ una mappa continua tale che $F(t, s) = \sigma(t)$ per ogni $s \in I$. \nl
        A questo punto considerando la catena singolare $c = F \circ [e_0 e_1 e_1'] + F \circ [e_0 e_1' e_0']$ si verifica che $\partial{c} = \sigma + \sigma^{-1}$. Dunque $[\sigma^{-1}] = -[\sigma] +$ in $H_1(X)$.
        \item Infine vedere dalle dispense di Simone Saccani come si mostra che $[\sigma * \tau] = [\sigma] + [\tau]$ in $H_1(X)$.
    \end{enumerate}
\end{remark}

\begin{theorem} [Relazione tra $H_1(X)$ e l'abelianizzato di $\pi_1(X, x_0)$] \nl
    Sia $X$ spazio topologico con base puntata $x_0 \in X$. \nl
    Sia può definire $h': \pi_1(X, x_0) \to H_1(X)$ come:
    \[
        h'([\sigma]) = [\sigma]
    \]
    Poiché $H_1(X)$ è abeliano, $h'$ fattorizza attraverso l'abelianizzato di $\pi_1(X, x_0)$ dando luogo a un omomorfismo di gruppi:
    \[
        h: \pi_1(X, x_0)^{ab} \to H_1(X)
    \]
    Se $X$ è connesso per archi allora $h$ è un isomorfismo. Dunque:
    \[
        H_1(X) \cong \pi_1(X, x_0)^{ab}
    \]
\end{theorem}

\begin{proof}
    Si vuole costruire una inversa $l: H_1(X) \to \pi_1(X, x_0)^{ab}$. \nl
    Per ogni punto $x \in X$ indichiamo con $u_x$ un cammino da $x_0$ a $x$. \nl
    Definiamo 
    \[
        l': \begin{cases}
            C_1(X) \to \pi_1(X, x_0)^{ab} \\
            \sigma \mapsto [u_{\sigma(0)} * \sigma * u_{\sigma(1)}^{-1}]
        \end{cases}
    \]
    Per mostrare che $l'$ induce una mappa in omologia, dobbiamo verificare che $l'(\partial{c}) = 0$ per ogni $c \in C_2(X)$. \nl
    Un $2$-simplesso $\tau: \Delta^2 \to X$ definisce un'omotopia tra i cammini $\tau \circ [v_0 v_1]$, $\tau \circ [v_0 v_2]$ e $\tau \circ [v_1 v_2]$. \nl
    Si verifica che:
    \[
        l'(\partial{\tau}) = [u_{\tau(v_0)} * \tau \circ [v_1 v_2] * u_{\tau(v_2)}^{-1}] + [u_{\tau(v_0)} * \tau \circ [v_0 v_2] * u_{\tau(v_2)}^{-1}] + [u_{\tau(v_1)} * \tau \circ [v_0 v_1] * u_{\tau(v_0)}^{-1}] = 0
    \]
    Dunque $l'$ induce una mappa in omologia $l: H_1(X) \to \pi_1(X, x_0)^{ab}$. \nl
    Si verifica facilmente che $l$ e $h$ sono inverse l'una dell'altra.
\end{proof}

\begin{definition} [Escissione] \nl
    Siano $U \subset A \subset X$ e l'inclusione $i: (X \setminus U, A \setminus U) \to (X, A)$. \nl
    Si dice che $U$ può essere \textbf{escissa} da $(X, A)$ se l'omomorfismo indotto in omologia:
    \[
        i_*: H_q(X \setminus U, A \setminus U) \to H_q(X, A)
    \]
    è un isomorfismo per ogni $q \geq 0$.
\end{definition}

\begin{theorem}
    [Teorema di escissione] \nl
        Siano $A \subset X$ e $U \subset \mathrm{Int}(A)$. \nl
        Allora $U$ può essere escissa da $(X, A)$.
\end{theorem}

\begin{proof}
    Si definiscono due mappe di complessi di catene:
    \[
        \mathrm{sd}: C_q(X) \to C_q(X) \quad \text{e} \quad T: C_q(X) \to C_{q+1}(X)
    \] 
    osservando che per funtorialità basta definire $\mathrm{sd}(\mathrm{id}_{\Delta^q})$ e $T(\mathrm{id}_{\Delta^q})$. \nl
    Sia $B_q$ il baricentro di $\Delta^q$, e se $\sigma = [v_0 \cdots v_{q-1}]$ un $(q-1)$-simplesso allora $B_q \sigma = [B_q v_0 \cdots v_{q-1}]$ è il $q$-simplesso ottenuto aggiungendo il baricentro come vertice. \nl
    Si definiscono $\mathrm{sd}(\mathrm{id}_{\Delta^q})$ e $T(\mathrm{id}_{\Delta^q})$
    \[
        \mathrm{sd}(\mathrm{id}_{\Delta^q}) = \begin{cases}
            \mathrm{id}_{\Delta^0} & q = 0 \\
            B_q \mathrm{sd}(\partial{\Delta^q}) & q > 0
        \end{cases}
    \]
    \[
        T(\mathrm{id}_{\Delta^q}) = \begin{cases}
            0 & q = 0 \\
            B_q(\mathrm{id}_{\Delta^q} - \mathrm{sd}(\partial{\Delta^q}) - T(\partial{\Delta^q})) & q > 0
        \end{cases}
    \]
    Si verifica che valgono le seguenti proprietà:
    La suddivisione è un morfismo di complessi di catene:
    \[
        \partial_q \circ \mathrm{sd} = \mathrm{sd} \circ \partial_q
    \]
    mentre $T$ è un'omotopia di catene tra l'identità e la suddivisione:
    \[
        \partial_{q+1} \circ T_q + T_{q-1} \circ \partial_q = \mathrm{id}_{C_q(X)} - \mathrm{sd}_q
    \]
    Dunque $\mathrm{sd}_* = \mathrm{id}_*$ in omologia. \nl
\end{proof}

\end{document}